\subsection{Equity Valuation Application and Processes}

\begin{definition} \hlt{Intrinsic Value}\\
Valuation of an asset or security with complete understanding of the investment characteristic.
\end{definition}

\begin{definition} \hlt{Mis-pricing}\\
Difference between estimated intrinsic value and the market value of an asset.\\
Mis-pricing can be broken into true mis-pricing, and error in estimate of intrinsic value.
\begin{equation}
V_E - P = (V - P) + (V_E - V) \nonumber
\end{equation}
where $V_E$ is estimated intrinsic value, $V$ is actual intrinsic value, $P$ is market price.
\end{definition}

\begin{definition} \hlt{Going concern assumption} assumes that the company will continue to operate as a business.\\
\hlt{Going-concern value} of the company is its value under a going-concern assumption.\\
\hlt{Liquidation value} is value of the company if assets of the firm are sold separately, net company assets.
\end{definition}

\begin{definition} \hlt{Fair market value} is the price that a willing, informed, able seller will trade an asset to a similar buyer. Company's market price should reflect its fair market value over time if market has confidence that the management is acting in interest of equity investors.\\
\hlt{Investment value} is the value of a stock to a particular buyer, and is dependent on buyer's specific needs and expectations, and perceived synergies with existing buyer assets.\\
For most investment decisions, intrinsic value is relevant. For acquisitions, investment value is more appropriate.
\end{definition}

\begin{remark} The applications of equity valuation are as follows:
\begin{enumerate}[label=\roman*.]
\setlength{\itemsep}{0pt}
\item Stock selection: whether the security is fairly priced relative to estimated intrinsic value and comparables
\item Inferring market expectation: evaluate reasonableness of expectations implied by market price by comparing with market implied expectation. Market expectation for fundamentals may be useful as benchmark or comparison value of the same characteristic for another company.\\
To analyst market expectation, select a valuation model that relates value to fundamental expectations and is appropriate given characteristics of the stock. Next, estimate values for all fundamentals in the model except the fundamental of interest. Lastly, solve for the value of the fundamental of interest that results in the model value equal to the current market price.
\item Evaluating corporate events: assess how the events affect a company's cash flow and equity value. Also, in M\&A, as acquirer common stock is often used for purchase, to know if the stock is fairly valued.
\item Rendering fairness opinions: parties involved in merger may be required to seek a fairness opinion on terms of the merger from a third party, which will use valuation.
\item Evaluating business strategies and models: companies maximising shareholder value will evaluate effect of alternative strategies on share value.
\item Communication with analyst and investors: valuation provides target audience with a common basis to discuss and evaluate performance, current state, and future plans.
\item Appraising private businesses: to determine value of firms for transaction purposes and tax-reporting purposes, as well as to evaluate firm characteristics for IPOs.
\item Share-based payment: valuation for executive compensation.
\end{enumerate}
\end{remark}

\subsubsection{Valuation Process}

\begin{method} \hlt{Valuation Process Steps}
\begin{enumerate}[label=\arabic*.]
\setlength{\itemsep}{0pt}
\item Understanding the business: industry and competitive analysis, analysis of financial statements and other company disclosures.
\item Forecasting company performance: via forecast of sales, earnings, dividends, financial position (pro forma)
\item Selecting appropriate valuation model: depending on characteristics of company and context of valuation
\item Converting forecasts to a valuation: estimating value involves judgement
\item Applying valuation conclusions: for investment recommendations, provide opinion on price of transaction, or evaluate economic merits of potential strategic investment.
\end{enumerate}
\end{method}

\begin{remark} \hlt{Elements of Industry Structure} by Porter's Five Forces:
\begin{enumerate}[label=\arabic*.]
\setlength{\itemsep}{0pt}
\item Threat of new entrants
\item Threat of substitutes
\item Bargaining power of buyers
\item Bargaining power of suppliers
\item Rivalry among existing competitors
\end{enumerate}
Attractiveness (long-term profitability) is determined by interaction of these five competitive forces.
\end{remark}

\begin{remark} \hlt{Generic Strategies for Competing}
\begin{enumerate}[label=\roman*.]
\setlength{\itemsep}{0pt}
\item Cost leadership: being the lowest cost producer while offering comparable products.
\item Product differentiation: addition of product features or services to increase attractiveness of the firm's product so that it will command a price premium.
\item Focus: seeking a competitive advantage within a target segment or segments of the industry either through cost leadership or differentiation.
\end{enumerate}
Once a strategy has been identified, evaluate the performance of the business on how well it executes.
\end{remark}

\begin{flushleft}
Quality of Earnings Indicators
\begin{tabularx}{\textwidth}{p{7em}|p{18em}|X}
\hline
\rowcolor{gray!30}
Category & Observation & Potential Interpretation \\
\hline 
Revenue, gains & 
\xxx Recognising revenue early &
Boosts reported income, making a decline in operating performance \\
& 
\xxx Classification of non-operating income or gains as part of operations &
Income/gains non-recurring, not relate to true operating performance, masking declines in operating performance \\
\hline
Expenses, losses &
\xxx Recognising too much or too little reserves in current year &
Boost current income at expense of future income (or vice versa) \\
& 
\xxx Deferral of expenses by capitalising expenditures as asset &
Boost current income at expense of future income. Mask problems with underlying business performance \\
&
\xxx Aggressive estimates and assumptions &
May indicate actions taken to boost income. Changes in assumptions indicate attempt to mask problems with underlying performance in current period.\\
\hline
BS issues &
\xxx Use of off-BS financing, i..e, securitising receivables &
Assets and/or liabilities not properly reflected on BS \\
\hline
CFO &
\xxx Characterisation of an increase in bank overdraft as operating cash flow &
Operating cash flow artificially inflated \\
\hline
\end{tabularx}
\end{flushleft}

\begin{remark} \hlt{Warning Signs of Poor Earnings Quality}
\begin{enumerate}[label=\roman*.]
\setlength{\itemsep}{0pt}
\item Excessive pressure on employees to meet revenue target, combined with dominant, aggressive management
\item Management/director compensation tied to profitability or stock price
\item Economic, industry, or company-specific pressures on profitability
\item Management pressure to meet debt covenants or earnings expectations
\item Existence of related-party transactions
\item Complex organisational structure, where party in control is not apparent
\item High turnover of management, directors, or legal counsel
\item Reported (through regulatory filings) disputes with and/or changes in auditors
\item History of securities law violations, reporting violations, or persistent late filings
\end{enumerate}
\end{remark}

\begin{method} \hlt{Absolute Valuation Models}\\
Estimates asset intrinsic value based on investment characteristics.
\begin{enumerate}[label=\roman*.]
\setlength{\itemsep}{0pt}
\item Dividend Discount Models: estimate value of shares based on present value of all expected dividends discounted at opportunity cost of capital. Measure of cash flow may be expanded to include all expected cash flow to firm not payable to senior claims. These include free cash flow and residual income approach.
\item Asset-Based Models: estimates value of firm as sum of market value of assets it owns or controls. Used to value firms that own or control natural resources.
\end{enumerate}
\end{method}

\begin{method} \hlt{Relative Valuation Models}\\
Estimates asset value relative to other assets. Most common models use market price as a multiple of an individual financial factor, such as P/E ratio.
\end{method}

\begin{method} \hlt{Sum-of-Parts Valuation}\\
Value individual parts of a firm and add them up. Useful when company operates multiple divisions (or product lines) with different business models and risk characteristics (conglomerate).	
\end{method}

\begin{remark} \hlt{Conglomerate Discounts}\\
To markdown value of company that operates in multiple unrelated industries, as compared to value of company that has a single industry focus.
\begin{enumerate}[label=\roman*.]
\setlength{\itemsep}{0pt}
\item Internal Capital Inefficiency: company allocation of capital to different divisions may not have been based on sound decisions.
\item Endogenous (Internal) Factors: company may have pursued unrelated business acquisitions to hide poor operating performance
\item Research Measurement Errors: discount do not exist, and is a result of incorrect measurement
\end{enumerate}
\end{remark}

\begin{remark} \hlt{Criteria for Model Selection}
\begin{enumerate}[label=\roman*.]
\setlength{\itemsep}{0pt}
\item Consistent with characteristics of the company being valued
\item Appropriate given the availability and quality of data
\item Consistent with purpose of valuation, including analyst's perspective
\end{enumerate}
Multiple models are often used, and examining differences in estimated value can reveal how a model's assumption and perspective of the analysis affect the estimated values.
\end{remark}

\begin{flushleft}
Format for Research Reports
\begin{tabularx}{\textwidth}{p{10em}|X|X}
\hline
\rowcolor{gray!30}
Section & Purpose & Content \\
\hline 
Table of Contents &
\xxx Show report organisation, typically in long reports &
\xxx Consistent with narrative in sequence and language \\
\hline
Summary, Investment Conclusion &
\xxx Communicate large picture
\xxx Communicate major conclusions 
\xxx Recommend course of action &
\xxx Capsule description of company
\xxx Major recent developments
\xxx Earnings projections
\xxx Other major conclusions
\xxx Valuation summary
\xxx Investment action \\
\hline
Business Summary &
\xxx Present company in more detail
\xxx Communicate detailed understanding of company economics, situation
\xxx Provide and explain specific forecasts &
\xxx Description to divisional level
\xxx Industry analysis
\xxx Competitive analysis
\xxx Historical performance
\xxx Financial forecasts \\
\hline
Risks &
\xxx Alert readers to risk factors in investing in the security &
\xxx Negative industry developments
\xxx Negative regulatory, legal risks
\xxx Negative company developments
\xxx Risks in the forecasts
\xxx Other risks \\
\hline
Valuation & 
\xxx Communicate a clear and careful valuation &
\xxx Description of model used
\xxx Recapitulation of inputs
\xxx Statement of conclusions \\
\hline
Historical, Pro Forma Tables &
\xxx Organise and present data to support analysis in Business Summary\\
\hline
\end{tabularx}
\end{flushleft}
