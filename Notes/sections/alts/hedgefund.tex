\subsection{Hedge Fund Strategies}

\subsubsection{Overview of Hedge Fund Strategies}

\begin{remark} \hlt{Classification of Hedge Funds and Strategies}
\begin{enumerate}[label=\roman*.]
\setlength{\itemsep}{0pt}
\item Equity: focus primarily on equity markets, with primarily equity risk. Includes long-short equity, dedicated short bias, equity market neutral.
\item Event-Driven: focuses on corporate actions such as governance events, mergers and acquisitions, bankruptcy etc. Soft-catalyst is investment made before event announcement; hard-catalyst is investment made after event announcement. Primarily takes on event risk. Includes merger arbitrage, distressed securities etc.
\item Relative Value: focused on relative valuation between two or more securities. Primarily takes on credit and liquidity risk. Includes fixed-income arbitrage, convertible bond arbitrage etc.
\item Opportunistic: top-down approach, focusing on multi-asset (macro-oriented) opportunity set. Risk depends on opportunity set, vary across time and asst class. Includes global macro, managed futures etc.
\item Specialist: focus on special or niche opportunities which require a specialised skill or knowledge of specific market. Exposed to unique risks from particular market sectors, niche securities, and/or esoteric instruments. Includes option volatility strategies, reinsurance strategies.
\item Multi-Manager: focus on portfolio of diversified hedge fund strategies. Managers combine diverse strategies and dynamically re-allocate over time. Includes multi-strategy funds and funds-of-funds.
\end{enumerate}
\end{remark}

\begin{remark} \hlt{Long-Short Equity}\\
Long positions believing to rise in value, short positions believing to fall in value. May shift between industry sectors, factors, and geography. Strong focus on fundamental research. Not known for market-timing skills from portfolio allocation perspective. To take concentrated positions if there is high conviction, and may apply leverage to increase positions.
\begin{enumerate}[label=\roman*.]
\setlength{\itemsep}{0pt}
\item Investment Characteristics: resulting portfolio has beta equal to weighted sum of positive and negative betas of various long-short positions. Manager does not seek to eliminate market exposure entirely, but with $40\% \sim 60\%$ net long position, which is beneficial as markets generally trend upward. May seek to provide returns comparable to long-only funds, but with half the standard deviation.
\item Strategy Implementation: managers may specialise in specific geography, sector, or investment style; may also generalise with a more balanced and flexible approach, but may miss detailed industry subtleties. Short positions may be used as hedge against unexpected market downturns, or used opportunistically when uncovering negative issues from a company. Performance in crisis may differentiate managers.\\
Some managers may add alpha via market timing of portfolio beta tilt, but most managers do this poorly. Index funds may be used to achieve desired exposure.\\
Comparatively market neutral, may need leverage to achieve returns.
\item Role in Portfolio: derive alpha from long-short single-stocks and naturally net long embedded beta. To weight if investment is worthwhile given potentially high fees (as long-only equity position may be more efficient to achieve comparable beta exposure).
\end{enumerate}
\end{remark}

\begin{remark} \hlt{Dedicated Short-Selling and Short-Biased Equity}\\
Dedicated short-selling managers take short-on positions that are overpriced. Short-biased managers use similar strategy, except short position is slightly offset by long exposure.\\
Activist short-selling managers not only takes short position, but also presents research that contends that the stock is overpriced. Not deemed market manipulation as long as activist short-seller is not publishing erroneous information, not charging for such information, and is acting in best interest of limited partners.
\begin{enumerate}[label=\roman*.]
\setlength{\itemsep}{0pt}
\item Investment Characteristics: managers eek negative correlation with conventional securities. Return is typically less compared to other hedge fund strategies, but with negative correlation benefit.\\
As markets generally rise over time, there is tendency for negative return for shorts.\\
More volatile than long-short equity. Limited partnership used due to difficult operational aspects.
\item Strategy Implementation: dedicated short-seller only trade short, but may moderate short beta with cash, with $60\% \sim 120\%$ short. Short-biased moderate short beta with some value-oriented long exposure or index-oriented long exposure as well as cash, with $30\% \sim 60\%$ short.\\
May face regulatory challenges as countries may impose limit or stringent rules on short selling. Limits may also be placed during extreme market environments.\\
Return profile for successful manager is increasingly positive returns as market declines, and risk-free rate when market rises. Possible to generate positive returns even as the market trends up by looking for short-selling targets from overvalued or mismanaged companies, or in-development single products.\\
Sufficient natural volatility means no need to add much leverage.
\item Role in Portfolio: produce returns uncorrelated or negatively correlated to conventional assets, provide diversification to portfolio. However, expected returns are relatively low.
\end{enumerate}
\end{remark}

\begin{remark} \hlt{Equity Market Neutral}\\
Manager takes long-short positions in similar or related equities with divergent valuations, with near-zero beta.\\
Manager may also set betas for sectors or industries as well as common risk factors equal to zero.
\begin{enumerate}[label=\roman*.]
\setlength{\itemsep}{0pt}
\item Investment Characteristics: create portfolio that generates alpha and is relatively immune to movements in overall market. Returns are relatively modest; leverage must be applied for meaningful return.\\
Funds may offer significant diversification and low volatility.
\item Strategy Implementation: long stocks that are temporarily undervalued, short stocks that are temporarily overvalued. In mean reversion, alpha will be created.\\
Portfolios constructed using highly quantitative methodologies, with diverse holdings and short holding time. Zero market beta may be constructed with derivatives.\\
Subtypes of funds includes:
\begin{enumerate}[label=\arabic*.]
\setlength{\itemsep}{0pt}
\item Pairs Trading: two stocks with similar characteristics but are overvalued and undervalued are identified. Co-integrated trading prices of pairs are monitored, and unusual divergence exploited.
\item Stub Trading: going long and short shares of a subsidiary and its parent. Positions taken correspond to percentage of subsidiary owned by parent.
\item Multi-Class Trading: going long and short relatively mis-priced share classes of same firm (i.e., non-voting and voting). Alpha from reversion of prices of shares to historical relative valuations.
\end{enumerate}
\item Role in Portfolio: produce alpha without taking market beta risk. Composition of portfolio creates less volatility than funds that rely on beta as source of return. Used when markets are volatile and poor.
\end{enumerate}
\end{remark}

\begin{remark} \hlt{Merger Arbitrage}\\
Generate return from uncertainty that exists in market during time between announcement of acquisition to completion of acquisition. Profit by correctly anticipating the outcome of various deals; analogous to writing insurance on an acquisition, where if acquisition is completed, manager earns a premium.
\begin{enumerate}[label=\roman*.]
\setlength{\itemsep}{0pt}
\item Investment Characteristics: return is typically $3\% \sim 7\%$ spread depending on deal-specific risks. If average time for deal completion is $3$ to $4$ months, with capital recycled into new deals and with $3 \times \sim 5 \times$ leverage, net annualised returns may reach $7\% \sim 12\%$, with little correlation to non-deal-specific factors.\\
If deal fails, initial price rise is typically reversed, with up to $40\%$ loss; strategy has left-tail risk.\\
Compared to other strategies, merger arbitrage is more liquid.
\item Strategy Implementation: for cash-for-stock acquisition, leverage may be used to buy target firm. On stock-for-stock acquisition, to long target and short acquirer, but shorting may be difficult due to liquidity issues (i.e., emerging markets). Derivatives used to overcome short-sale constraints and manage risks.\\
Convertibles may be used to provide cushion if deal fails. If acquirer credit is superior to target, CDS may be used, with short target to benefit from improved credit quality after merger. Long CDS may be used as partial hedge against merger failing. Overall market risk hedged with short equity index ETFs/Futures or long equity index put positions.
\item Role in Portfolio: sharpe ratios are high, as strategy usually produce relatively steady returns. However, there is significant left-tail risk.
\end{enumerate}
\end{remark}

\begin{remark} \hlt{Distressed Securities}\\
Take positions in companies that are in bankruptcy, facing potential bankruptcy, or under financial stress, due to waning competitiveness, excessive leverage, poor governance, accounting irregularities, fraud.\\
Securities often trade at greatly depressed prices. As some institutions such as insurance and banks are not permitted to hold non-investment-grade securities, there may be significant pricing inefficiencies for profit.\\
Firm may be liquidated, or reorganised. If reorganised, debtors may exchange debt for new equity or agree to extension of maturity.
\begin{enumerate}[label=\roman*.]
\setlength{\itemsep}{0pt}
\item Investment Characteristics: returns are greater than other event-driven strategies, but with larger variability of outcomes. Lock-up periods is comparatively long (no redemptions for first two years), reflecting extended periods of time it can take to value and exit a distressed security.\\
Returns are lumpy and cyclical. Attractive in early stage of economic recovery after financial crisis.
\item Strategy Implementation: some managers take passive position; some will acquire majority of certain class of security to take creditor control in bankruptcy.\\
Successful execution requires skillset in analysing complicated legal proceedings, bankruptcy processes, creditor committee discussions, re-organisation scenarios, anticipate market reactions to corporate actions.\\
'Capital structure arbitrage' may be captured, by long securities expected to receive recoveries, short securities where value-recovery prospect are dim.\\
In an active negotiation process, fulcrum securities (partially-in-the-money claims) identified to provide leverage in re-organisation.\\
Investment style is usually long-biased, with high levels of illiquidity in concentrated activist approach.\\
Moderate to low leverage is used, typically with $1.2 \times \sim 1.7 \times$ NAV used, and some nominal leverage from derivatives hedging.
\item Role in Portfolio: strategy involves moderately high levels of illiquidity due to nature of assets purchased. Returns are on-average higher than other event-driven strategies, but unpredictable and sensitive to declines in overall market.
\end{enumerate}
\end{remark}

\begin{remark} \hlt{Fixed-Income Arbitrage}\\
Buy relatively undervalued securities and short relatively overvalued securities with expectation that mis-pricing will resolve itself within specified investment horizon. Valuation differences beyond normal historical range may be due to differences in credit quality, liquidity, volatility, issue sizes.\\
Realising net positive relative carry may also be achieved through exploiting kinks in yield curve or an expected shift in shape of yield curve.
\begin{enumerate}[label=\roman*.]
\setlength{\itemsep}{0pt}
\item Investment Characteristics: as pricing inefficients are quite small, but correlation aspects across different securities is quite high, substantial leverage is used to exploit inefficiencies, with leverage ratio of $4 \times \sim 15 \times$.\\
Liquidity of fixed-income arbitrage depends on particular strategy employed and kinds of fixed-income instruments used, i.e., US Treasures are very liquid, while MBS or foreign instruments are less liquid.
\item Strategy Implementation: long-short positions may reduce interest rate risk by taking duration-neutral positions, providing hedge against small shifts in yield curves. To hedge against large yield changes and/or non-parallel yield curve movements, fixed-income derivatives may be used. In addition, derivatives are used to hedge sovereign risk, credit risk, pre-payment risks etc depending on portfolio.
\begin{enumerate}[label=\arabic*.]
\setlength{\itemsep}{0pt}
\item Yield Curve Trades: calendar spread strategy involves long-short positions at different points of yield curve to profit from anticipated yield curve steepening or flattening. If positions are taken in securities of different firms, then liquidity, credit, and interest rate risks will be present. Otherwise, interest rate movements would be main source of risk.
\item Carry Trades: short low-yield, long high-yield. Source of return is from yield differential (as spreads narrow) and price changes as mean reversion occurs.
\end{enumerate}
\item Role in Portfolio: return distribution similar to returns from writing puts. If trade follows expectations, earn return from spread narrowing and positive carry; else use of leverage may result in negative returns.\\
Drawback is that highly leveraged nature may cause modest price volatility and lead to domino effect of margin calls and de-leveraging.
\end{enumerate}
\end{remark}

\begin{remark} \hlt{Convertible Bond Arbitrage}\\
Profit from purchasing implied volatility of convertible bonds, which is often underpriced. Take other positions to hedge the delta and gamma risk of convertible bond.
\begin{enumerate}[label=\roman*.]
\setlength{\itemsep}{0pt}
\item Investment Characteristics: convertible bonds are impacted by overall interest rate levels, corporate credit spreads, bond coupon and principal cash flows, value of embedded stock option etc. Most convertible bonds are non-rated, thinly-traded, hence the embedded options trade at relatively low implied volatility levels compared to historical volatility level of underlying security. Convertibles are cyclical relative to amount of new issuances of such securities; the higher the new issuance, the cheaper their pricing and hence more attractive the arbitrage opportunities available.\\
To extract the cheap embedded optionality, ned to hedge away interest rate risk, credit risk, market risk with combination of interest rate derivatives, CDS, short sales of appropriate delta-adjusted amount of underlying or purchase of stock options, which may erode attractiveness of the asset.\\
Credit-oriented convertible managers may choose not to hedge the credit default risk. Volatility-oriented managers may over-hedge equity risk to create bearish tilt on underlying with higher volatility exposure.
\item Strategy Implementation: long relatively undervalued convertible bond, short relatively overvalued underlying stock. Strategy profits from sufficiently large stock price swings and proper periodic balancing.\\
If realised equity volatility exceeds implied volatility of embedded option, gain is achieved.\\
Short selling may be vulnerable to short squeeze. Credit issues may complicate valuation when credit spreads widen or narrow. Time decay of embedded call option in reduced volatility periods lead to losses.\\
Relatively high levels of leverage required to extract modest ultimate gain from delta hedging, with portfolios at $300\%$ long vs $200\%$ short, the short exposure being function of delta-adjusted equity.
\item Role in Portfolio: strategy perform best when convertible issuance is high, general volatility levels are moderate, liquidity to trade and adjust positions is ample.\\
Strategy may not perform well in periods of illiquidity or acute credit weakness.
\end{enumerate}
\end{remark}

\begin{remark} \hlt{Global Macro Strategies}\\
Attempts to make correct assessments and forecasts of global economic variables such as inflation, currency exchange rates, yield curves, central bank policies, general economic health of various countries. Broad range of security types and global asset classes including derivatives are used to take positions on these views.\\
Manager tend to focus on certain themes, regions, or styles.
\begin{enumerate}[label=\roman*.]
\setlength{\itemsep}{0pt}
\item Investment Characteristics: managers may take directional or thematic positions. Leverage through use of derivatives may be used to magnify profits, with margin-to-equity ratio of $15\% \sim 25\%$ posted against futures or forward positions. Source of returns from correctly discerning and capitalising global trends.\\
Mean-reverting low volatility markets are not favourable for global macro returns.\\
If global economies do not behave as expected or unanticipated risks emerge, returns may be uneven and volatile. However, strategy is useful over full market cycle for diversification.
\item Strategy Implementation: based on top-down analysis. Different managers may implement strategies with different techniques, i.e., technical analysis vs fundamental analysis, or discretionary vs systematic.\\
Manager making directional predictions will generally use fundamental analysis to determine if asset is overvalued or undervalued. Manager using relative value strategy will consider securities which are under- or overvalued relative to each other.
\item Role in Portfolio: strategy may add portfolio diversification to portfolio of traditional assets. Manager will anticipate changes before other market participants and take corresponding position before market reacts. This contrarian tendency makes allocation to global macro managers advantageous.\\
In times of market stress, managers have historically delivered right-tail skewed returns.
\end{enumerate}
\end{remark}

\begin{remark} \hlt{Managed Futures Strategy}\\
Long-short positions on variety of derivatives contracts such as futures, options on futures, forwards and swaps, commodities and currencies, and even more exotic contracts.
\begin{enumerate}[label=\roman*.]
\setlength{\itemsep}{0pt}
\item Investment Characteristics: positive skewness in periods of market stress, but return profile is very cyclical due to mean-reverting characteristics of markets, and is more volatile.\\
Managers may apply great amounts of leverage, with $1/8$ of capital as collateral on futures, and rest of capital in some highly liquid security that serve as collateral for futures clearinghouse.\\
The funds are highly liquid, as the futures are highly liquid; they trade globally and continuous. Long-short positions allow manager to easily access exposures across range of asset classes.\\
Crowding has occured, as many market participants pursue same trades and use similar signals.
\item Strategy Implementation: involve a momentum/trend driven trigger or volatility signal across different time horizons, with short-term mean reversion filters. The longer-term model will have higher weight than short-term filter. Position sizing considers volatility of each underlying futures position as well as correlation of returns. The greater the volatility, the smaller the sizing.\\
For closing a position exit methodologies will be a price target, momentum reversal, time-based exit, trailing stop-loss, or some combination thereof.\\
To develop rules and signals that performs well in future out-of-sample period. As more managers use similar signals, there will be alpha decay.
\begin{enumerate}[label=\arabic*.]
\setlength{\itemsep}{0pt}
\item Time-Series Momentum: follow the trend, long assets increasing in price, short assets falling in price. May be net long or net short; work best when past returns are good predictor of future returns.
\item Cross-Sectional Momentum: implemented with a cross-section of assets, generally within an asset class. Long those rising in price, short those falling in price. Result in net zero or market-neutral position. Work well when market's relative performance to other markets is reliable predictor of future performance. May be constrained by limited futures contracts.
\end{enumerate}
\item Role in Portfolio: managed futures have very little correlation with traditional equity and fixed-income assets. Hence, this will generally improve the total risk-adjusted return. Diversification is used in times of market stress, where right-skewed return distribution provides significant advantage.
\end{enumerate}
\end{remark}

\begin{remark} \hlt{Volatility Trading}\\
Relative value volatility arbitrage strategy goal is to source and buy cheap volatility and sell more expensive volatility while netting out time decay aspects normally associated with option portfolios. Depending on instruments used, value may be extracted from active gamma trading adjustments when markets move.\\
Generally, volatility prices in Asian markets is lower than in other regions.\\
Another volatility trade involves acting as counterparty to market participants consistently seeking long volatility. Due to negative correlation between stock market returns and equity volatility, volatility may be used as a hedge. Selling volatility will earn an insurance-like premium; upturn in volatility may unravel the strategy.\\
VIX futures tend to be mean reverting as high volatility naturally tends to dissipate over time.
\begin{enumerate}[label=\roman*.]
\setlength{\itemsep}{0pt}
\item Investment Characteristics: long volatility has positive convexity, useful for hedging. On short side, option premium sellers extract steadier returns in normal market environments. Relative value trading may provide alpha across different geographies and asset classes. Liquidity varies across different instruments.\\
Use of futures contracts makes it easy to apply leverage. Convexity of volatility derivatives mean large gains can be made going long volatility while taking little risk.\\
Volatility trading benchmarking is difficult as it is a niche strategy.
\item Strategy Implementation:
\begin{enumerate}[label=\arabic*.]
\setlength{\itemsep}{0pt}
\item Time-Zone Arbitrage: capturing volatility spread between options traded in Asian time zones against those traded in London, New York, or Chicago
\item Cross-Asset Volatility Trading: sourcing asian market implied volatility at cheaper levels than US market implied volatility, even though asian market option has realised volatility higher than that of the US market option.
\item Long Volatility Trading: equity volatility is $\sim 80\%$ negatively correlated with equity market returns, making long volatility strategy potential diversifier for long equity strategies.
\item Option Strategies: bull and bear spreads, straddles, calendar spreads may be constructed using basic exchange-traded options, which captures relative timing and strike pricing opportunities
\item Over-the-Counter (OTC) Options: tenor and strike prices of options customised. Subject to counterparty risk and illiquidity risk.
\item Futures on VIX Index: directly express view on volatility without need for hedging. As VIX index is mean reverting, trend-following trades will be unprofitable. Abundant supply of traders and investors sell volatility and capture the premium and volatility roll down payoff, making profit difficult.
\item OTC Volatility Swap or Variance Swap: provides relatively pure exposure to volatility. Both swaps are forward contacts with payoff based on difference between observed variance and expected variance in contract, multiplied by notional amount.
\end{enumerate}
\item Role in Portfolio: long volatility may be diversifier as equity volatility is negatively correlated with market returns. However, premium must be paid to volatility seller.
\end{enumerate}
\end{remark}

\begin{remark} \hlt{Reinsurance/Life Settlements}\\
Insurance market has wide range of highly specific and detailed contracts which are less standardised. Hence, these are not as liquid. Hedge funds may have differentiated view of individual or group life expectancy, which may provide attractive uncorrelated returns.\\
Reinsurance of catastrophe risk allows transfer of risk, capital management, and solvency management. These are also a source of uncorrelated return alpha.
\begin{enumerate}[label=\roman*.]
\setlength{\itemsep}{0pt}
\item Investment Characteristics: strategies are illiquid due to the unstandardised nature of insurance policies
\item Strategy Implementation: manager that invest in life settlements will analyse pools of life insurance contracts that brokers offer, and invest in those with attractive expected return. Profit if present value of future insurance payout exceeds present value of payments made by manager.\\
Manager will look for following policy characteristics:
\begin{enumerate}[label=\arabic*.]
\setlength{\itemsep}{0pt}
\item surrender value offered to insured individual is relatively low
\item ongoing premium payments to keep policy alive are also relatively low
\item probability is relatively high that designated insured person is likely to die relatively soon
\end{enumerate}
Appraising life settlement requires significant amount of skill and knowledge, and requires comparing individual policyholder's outlook to actuarial averages.\\
For catastrophe insurance investor, profit may be achieved if manager can
\begin{enumerate}[label=\arabic*.]
\setlength{\itemsep}{0pt}
\item obtain sufficient policy diversity in terms of geography and type of insurance offered
\item receive sufficient buffer in terms of loan loss reserves from insurance company
\item receive enough premium income
\end{enumerate}
Appraising catastrophe insurance requires using global weather patterns to make forecasts. These assumptions are weighted against reinsurance income to be received.
\item Role in Portfolio: risk inherent in these strategies is almost entirely uncorrelated with market risks ad business cycles. Hence managers may increase alpha wile adding diversification.
\end{enumerate}
\end{remark}

\begin{remark} \hlt{Fund of Funds (FoF)}\\
Managers aggregate investors' capital and investment in number of different funds.\\
Benefits include diversification across many hedge fund strategies, expertise in individual manager selection, strategic allocation and style allocation, due diligent, occasional value-added tactical decisions, currency hedging, leverage at portfolio level, better liquidity relative to individual hedge funds, access to certain closed hedge funds, economies of scale in fund monitoring, research expertise, potentially concessions from underlying funds.\\
Disadvantages include a second layer of fees for the investor, lack of transparency into individual hedge funds, no netting of performance fees, additional principal-agent problems.
\begin{enumerate}[label=\roman*.]
\setlength{\itemsep}{0pt}
\item Investment Characteristics: individual hedge funds use a $2$ and $20$ fee structure ($2\%$ management fee, $20\%$ performance incentive fees). FoF adds additional $1\%$ management fee and further $20\%$ incentive fee on total FoF portfolio (but over time, FoF fees become negotiable and smaller).\\
FoF makes investment into hedge funds practical for smaller investors, allowing diversified exposure to a number of individual funds, and less resources as due diligence will not be required on the underlying.\\
FoF also provides access to high-profile managers whose funds are otherwise closed to new investors. Larger size of FoF may allow the fund to obtain valuable concessions from underlying fund management.\\
FoF requires one-year initial lock-up before allowing greater liquidity afterwards. Underlying funds have stricter limits on liquidity which may put FoF manager in a squeeze.\\
Investors could be required to make substantial incentive payments to small number of successful underlying funds, even if overall performance of FoF is poor (netting risk).
\item Strategy Implementation: implemented as follows:
\begin{enumerate}[label=\arabic*.]
\setlength{\itemsep}{0pt}
\item Usage of fund database and personal introductions to become familiar with target hedge funds
\item Choose appropriate strategic allocation to different hedge fund strategies. Tactical allocation may also be conducted, where FoF will underweight or overweight various strategies to reflect FoF manager's perception of changing market environment.
\item Initiate manager selection process, applying both top-down and bottom-up techniques
\item For each hedge fund strategy, consider some candidates and interview managers
\item Review relevant materials such as audit reports, personnel, operational processes, risk management
\item Negotiate with managers for lower fees, improved liquidity, or other terms
\item After investment approval and inclusion in FoF, monitoring process begins, detecting major personnel changes, style drift etc.
\end{enumerate}
\item Role in Portfolio: greater diversification, steady returns, less concentrated exposure to risks, less volatility, less exposure to downside risk of any individual fund manager
\end{enumerate}
\end{remark}

\begin{remark} \hlt{Multi-Strategy Hedge Funds}\\
Multiple hedge fund strategies combined under same hedge fund structure.\\
Teams of managers dedicated to running different strategies share operational and risk management systems.
\begin{enumerate}[label=\roman*.]
\setlength{\itemsep}{0pt}
\item Investment Characteristics: diversification provides steady returns and low volatility.\\
Operational risks are not as well-diversified as that of FoF as all operational processes are under one roof.\\
Diversity of strategies are limited as managers in multi-strategy fund tend to have similar investment viewpoints and methods.\\
Tactical allocation decisions may be made with speed and ease due to high internal transparency and fast response time. Investor fees are more attractive than FoF, as investor only pays incentive fee on total fund performance, but some funds will use an FoF 'pass-through' fee model (resulting in netting risk).\\
Investor liquidity is limited using redemption periods and initial lockups. There is additional limits on amount of redemption each quarter.
\item Strategy Implementation: investments are made in number of varying strategies.\\
Tactical allocation may be made with speed and ease; internal teams are well informed on why and when capital and leverage should be reallocated across strategies, while FoF is more opaque in the process.\\
Risk management may be more effective as managers have solid understanding of correlation and common risks between various funds.\\
There is efficiency from sharing same administrative resources.\\
There is greater use of leverage than FoF. In periods of market stress, left-tail risks may threaten survival of fund. Multi-strategy funds have more varied performance than FoF.
\item Role in Portfolio: adds diversification and steady, low-volatility returns. Performs better than FoF due to superior fee structure, greater ability to execute tactical asset allocation. Leveraged nature may lead to left-tail blow-up in times of stress.
\end{enumerate}
\end{remark}

\subsubsection{Analysis of Hedge Fund Strategies}

\begin{remark} \hlt{Conditional Factor Risk Model}\\
Conditional model can show whether hedge fund exposures that are insignificant in calm market periods may become significant in turbulent market periods.
\begin{equation}
r_{\text{Hedge Fund}, i, t} = \alpha_i + \sum\limits_{n=1}^n \beta_{i,n} (\text{Factor } n)_t +  \sum\limits_{n=1}^n D_t \beta_{i,n} (\text{Factor } n)_t + \epsilon_{i,t} \nonumber
\end{equation}
where $\beta_{i,n} (\text{Factor } n)_t$ is exposure to risk factor $n$ for hedge fund $i$ at time $t$ in normal periods, $D_t \beta_{i,n} (\text{Factor } n)_t$ is incremental exposure to risk factor $n$ for hedge fund $i$ at time $t$ in turbulent market periods.\\
Any returns not explained by model's risk factor may be attributed to either omitted risk factors, alpha, or randomness. Hasanhodzic and Lo ($2007$) used the following factors:
\begin{enumerate}[label=\roman*.]
\setlength{\itemsep}{0pt}
\item Equity Risk (SNP500): monthly total return of S\&P $500$ index, including dividends
\item Interest Rate Risk (BOND): Bloomberg Barclays Corporate Intermediate Bond Index monthly return
\item Currency Risk (USD): monthly return of US Dollar index
\item Commodity Risk (CMDTY): monthly return of Goldman Sachs Commodity Index (GSCI)
\item Credit Risk (CREDIT): difference between monthly season Moody's Baa and Aaa corporate bond yields
\item Volatility Risk (VIX): first-difference of end-of-month value of CBOE Volatility Index (VIX)
\end{enumerate}
\end{remark}

\begin{method} \hlt{Stepwise Regression for Conditional Factor Risk Model}\\
The following process allows building of linear conditional factor models, avoiding multi-collinearity issues
\begin{enumerate}[label=\arabic*.]
\setlength{\itemsep}{0pt}
\item Identify potentially important risk factors
\item Calculate pairwise correlations across all risk factors. If two-state conditional models are used, calculate correlations across all risk factors for both states—for example, during normal market conditions (state 1) and during market crisis conditions (state 2). 
\item For highly correlated risk factors A and B, regress the return series of interest (i.e., hedge fund returns) on all risk factors excluding factor A. Then, regress the same returns on all the risk factors excluding factor B. Given the adjusted $R^2$ for regressions without A and without B, keep the risk factor that results in the highest adjusted $R^2$.
\item Repeat step 3 for all other highly correlated factor pairs, with the aim of eliminating the least useful (in terms of explanatory power) factors and thereby avoiding multi-collinearity issues.
\end{enumerate}
When run on Hasanhodzic and Loc ($2007$), BOND and CMDTY are dropped due to multicollinearity issues.
\end{method}

\begin{flushleft}
Interpretation of Conditional Risk Factor Exposures
\begin{tabularx}{\textwidth}{p{4em}|p{5em}|p{11em}|p{4em}|p{4em}|X}
\hline
\rowcolor{gray!30}
State & Risk Factor & Market Trend & Position & Factor & Comments \\
\hline
Normal & SNP500 & Equities rising & Long & Positive & Add risk, increase return \\
& Credit & Spreads flat/narrowing & Long & Positive & Add risk, increase return \\
& USD & USD flat/depreciating & Short & Negative & Sell USD, boost returns \\
& VIX & Volatility falling & Short & Negative & Sell volatility, boost returns \\
\hline
Crisis & DSNP500 & Equities falling sharply & Short & Negative & Aim to reduce risk \\
& DCREDIT & Spreads widening & Short & Negative & Aim to reduce risk \\
& DUSD & USD appreciating & Long & Positive & USD is haven in crisis periods \\
& DVIS & Volatility rising & Long & Positive & Negative corr with equities \\
\hline
\end{tabularx}
\end{flushleft}

\begin{remark} \hlt{Performance Contribution to a $60/40$ Portfolio}\\
Assuming $20\%$ allocation to at $60\%$ equity, $40\%$ bond to result in $20\%$ hedge fund, $48\%$ equity, $32\%$ bond,
\begin{enumerate}[label=\arabic*.]
\setlength{\itemsep}{0pt}
\item total portfolio standard deviation decreases
\item Sharpe ratio increases
\item Sortino ratio increases
\item Maximum drawdown increases (in $\sim 1/3$ of portfolios)
\end{enumerate}
Hedge fund strategies generally increase risk-adjusted return and provide diversification to a $60/40$ portfolio.
\end{remark}

\begin{remark} \hlt{Dissection of Return Performance on Traditional $60/40$ Portfolio}\\
High Sharpe ratio: systematic futures, distressed securities, FI arbitrage, global macro, equity market neutral.\\
High Sortino ratio: equity market neutral, systematic futures, long-short equity, event driven.\\
Superior Risk-Adjusted Performance: systematic futures, equity market neutral, global macro, event driven.\\
Do not significantly enhance risk-adjusted performance: FoF, multi-strategy
\end{remark}

\begin{remark} \hlt{Dissection of Risk Reduction on Traditional $60/40$ Portfolio}\\
Lowest standard deviation of returns: dedicated short-biased, bear market neutral\\
Low standard deviation of returns: systematic futures, FoF (macro, systematic), equity market neutral\\
The risk-reduction ability of these strategies are substantial. They also improve risk-adjusted returns.\\
Little impact on reducing standard deviation: event-driven (distressed), relative value (convertible arbitrage)\\
For event-driven distressed securities, lack of ability to reduce standard deviation cause these funds to take long positions in securities, result in either mild success or grand failures.\\
For relative value convertible arbitrage, strategy's leveraged nature becomes liability in times of volatility.
\end{remark}

\begin{remark} \hlt{Dissection of Drawdown on Traditional $60/40$ Portfolio}\\
Smallest maximum drawdowns: opportunistic strategies (global macro, systematic futures, merger arbitrage, equity market neutral)\\
By conditional risk model, these strategies have minimal exposure to credit risk or equity risk in market risk. These strategies also benefit from either liquid nature.\\
No effect on maximum drawdowns: long-short equity, event-driven (distressed), relative value (convertible arb)\\
By conditional risk model, these strategies have significant exposure to equity risk, and in crisis periods, they also have significant exposure to credit risk.
\end{remark}