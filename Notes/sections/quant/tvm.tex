\subsection{Time Value of Money}

\begin{definition}
\hlt{Expected Annual Rate}
\begin{align}
\text{EAR} &= (1 + \text{periodic rate})^m - 1 \nonumber \\
\text{EAR} &= e^r - 1 \nonumber
\end{align}
\end{definition}

\begin{definition}
\hlt{Continuous Compounding}
\begin{equation}
FV_N = PV e^{r_s N} \nonumber
\end{equation}
\end{definition}

\begin{definition}
\hlt{Ordinary Annuity}: first cash flow one period from now.
\begin{equation}
FV_N = A \left[ \frac{(1+r)^N - 1}{r} \right] \nonumber
\end{equation}
\end{definition}

\begin{definition}
\hlt{Annuity Due}: first cash flow occurs from today.
\begin{equation}
FV_N = A \left[ \frac{(1+r)^N - 1}{r} \right] (1+r) \nonumber
\end{equation}
\end{definition}

\begin{definition}
\hlt{Perpetuity}: never ending cash flows.
\begin{equation}
PV = \frac{A}{r} \nonumber
\end{equation}
\end{definition}

\begin{definition} \hlt{Internal Rate of Return (IRR)}\\
The interest rate at which a series of cash inflows and outflows sum to zero when discounted to present value.\\
Series of cash flows such that the net present value (NPV) is zero.
\end{definition}

\begin{definition} \hlt{Money-Weighted Rate of Return}\\
The IRR on a portfolio, taking into account all cash inflows and outflows.
\end{definition}

\begin{definition} \hlt{Time-Weighted Rate of Return}\\
Measures compound growth, rate at which $\$1$ compounds over specific performance horizon.
\begin{enumerate}[label=\arabic*.]
\setlength{\itemsep}{0pt}
\item Value the portfolio immediately preceding significant additions or withdrawals. Form sub-periods over the evaluation period that correspond to dates of deposits and withdrawals.
\item Compute the holding period return $HPR$ of the portfolio for each sub-period.
\item Compute the total return for the entire measurement period,
\begin{equation}
r = \sqrt[1/n]{\prod\limits_{i=1}^n (1+HPR_i)} - 1 \nonumber
\end{equation}
\end{enumerate}
\end{definition}

\begin{definition} \hlt{Annualised Returns}
\begin{equation}
r = (1+HPR)^{365/\text{Days}} - 1 \nonumber
\end{equation}
\end{definition}

\begin{definition} \hlt{Present Value of Cash Flow Given Frequent Compounding}
\begin{equation}
PV = FV_N \left(1 + \frac{r}{m} \right)^{-mN} \nonumber
\end{equation}
where $r$ is quoted annual interest rate, $N$ is number of years, $m$ is compounding periods per year.
\end{definition}

\begin{definition} \hlt{Continuously Compounded Return}\\
Given HPR, the associated continuously compounded return is
\begin{equation}
r_{\text{CC}} = \ln(1+HPR) = \ln \left(\frac{\text{Ending Value}}{\text{Beginning Value}} \right) \nonumber
\end{equation}
\end{definition}

\newpage

\begin{definition} {\color{white}space}
\begin{enumerate}[label=\roman*.]
\setlength{\itemsep}{0pt}
\item Gross Return: total return on security portfolio before deducing fees for management and admin fees
\item Net Return: return after management and admin fees have been deducted
\item Pre-Tax Nominal Return: return before paying taxes
\item After-Tax Nominal Return: return after tax liability is deducted
\item Real Return: nominal return after adjusting for inflation.
\begin{equation}
(1 + \text{Real Return}) = \frac{(1 + \text{Nominal Risk-Free Rate}) (1 + \text{Risk Premium})}{(1 + \text{Inflation Premium})} \nonumber
\end{equation}
\item Leveraged Return: return that is a multiple of return on underlying asset
\begin{equation}
r_{\text{Leveraged}} = \frac{r(V_0 + V_B) - r_B V_B}{V_0} \nonumber
\end{equation}
$r$ is un-levered rate of return, $V_0$ is un-levered amount, $r_B$ is cost of borrowing, $V_B$ is borrowed amount.
\end{enumerate}
\end{definition}
