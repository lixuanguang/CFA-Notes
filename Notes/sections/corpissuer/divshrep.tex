\subsection{Analysis of Dividends and Share Repurchases}

\begin{definition} Common Terminology
\begin{enumerate}[label=\roman*.]
\setlength{\itemsep}{0pt}
\item \hlt{Ex-Dividend Date}: first date that shares trade without right to receive declared dividend for the period.
\item \hlt{Payout Policy}: set of principles guiding cash dividends and value of shares repurchased
\end{enumerate}
\end{definition}

\begin{remark} Types of Dividends
\begin{enumerate}[label=\roman*.]
\setlength{\itemsep}{0pt}
\item \hlt{Regular Cash Dividends}: distribution of cash. Firms strive for stability by increasing them slowly, refraining from any reductions. Stable or increasing dividends are signs of consistent or increasing profitability.\\
Frequency of payment vary. US and Canadian companies typically pay quarter, European and Asian companies typically pay semiannually and annually.
\item \hlt{Extra or Special (Irregular) Dividends}: cash dividend supplementing regular dividends, or dividend from company that normally does not pay dividends. Paid under special circumstances with expectation that dividend is not recurring, or when company has a very profitable year but does not want to commit to higher ongoing regular dividend payment. Usually used by firms in cyclical industries.
\item \hlt{Liquidating Dividend}: paid when the company:
\begin{enumerate}[label=\arabic*.]
\setlength{\itemsep}{0pt}
\item is bankrupt, and net assets (after all liabilities are paid) are distributed to shareholders
\item sells a portion of its business for cash, proceeds distributed to shareholders
\item pays a dividend that exceeds accumulated retained earnings (reduces stated capital)
\end{enumerate}
Liquidating dividend is a return of capital, rather than a return on capital.
\item \hlt{Stock Dividend}: non-cash dividend paid in form of additional shares. After payment, shareholders have more shares, and cost per shares will be lower. Shareholder proportionate ownership does not change, as each shareholder receives same percentage of stock dividend. Stock dividends are not taxed. Market price per share declines, leaving shareholders with no net gain.\\
Stock dividends encourage long-term investing, hence may reduce cost of equity capital. Stock dividends also increase stock float and hence its liquidity, and decrease market price of stock to a desirable trading range that attracts more investors.\\
Companies paying same regular cash dividend per share has increased their share dividend, but companies that have the same payout ratio would decrease dividend per share (dividend yield unchanged).\\
Stock dividend is accounted for as transfer of retained earnings to contributed capital.
\item \hlt{Stock Splits}: similar to stock dividends but larger in size. Reverse stock splits reduce number of shares outstanding and increase price per share, used to increase market price of stock to a desirable range to attract institutional investors and mutual funds that shun low-priced stocks.
\end{enumerate}
\end{remark}

\begin{remark} \hlt{Effects of Dividends on Financial Statement and Ratios}\\
Cash dividend payments reduce cash and stockholders equity, resulting in lower quick ratio and current ratio, and higher leverage ratios (i.e., debt-to=equity, debt-to-asset).\\
Stock dividends, stock splits or reverse stock splits leave the capital structure unchanged, do not affect ratios.\\
For stock dividend, decrease in retained earnings (corresponding to value of stock dividend) is offset by increase in contributed capital, leaving value of total equity unchanged.\\
Stock split or reverse stock split does not affect book value of equity and tax cost basis for shareholders.
\end{remark}

\subsubsection{Theory of Dividend Policy}

\begin{remark} \hlt{Dividend Irrelevance Theory}\\
Based on Miller and Modigliani (MM), under perfect capital market assumptions (no taxes, transaction costs, equal information), dividend policy should have no impact on cost of capital or shareholder wealth. The MM theory is based on the firm's total payout policy.\\
Theory is based on concept of 'homemade dividends', where an investor may tailor his own dividends irrespective of company dividend policy. If cash dividend is too large, investor may take excess cash received and buy more stock. If cash dividend is too small, investor may sell some stock to get the cash flow. Hence, combination of value of investor investment in firm and cash in hand will be the same.\\
In real world, market imperfections causes issues, as companies issuing new shares incurs flotation costs. Shareholders selling shares would incur transaction costs and capital gain taxes.  Selling shares on periodic basis will be problematic if share prices are volatile.
\end{remark}

\begin{remark} \hlt{Dividend Preference Theory (Bird in Hand Argument)}\\
Even under perfect capital market assumptions, investors prefer dollar of dividends to dollar of potential capital gains from earnings as dividends are viewed as less risky.\\
Company that pays dividends will have lower cost of equity capital than a company that does not pay dividends, hence this should result in a higher share price.\\
When measuring total return, dividend yield component $\frac{D_1}{P_0}$ is less risky than growth component $g$.
\end{remark}

\begin{remark} \hlt{Tax Aversion Theory}\\
Investors prefer not to receive dividends due to higher tax rates. In the extreme, this implies investors would ant companies to have zero dividend payout ratio.\\
In real world, tax laws prevent companies from accumulating excess earnings; dividend payments necessary.
\end{remark}

\begin{remark} \hlt{Information Asymmetry}\\
Company board and management have more information compared to investors. Dividends convey credit information to investors, as dividends entail actual cash flow and are expected to continue in the future.\\
Companies avoid increasing dividends unless higher levels of dividends are expected to continue in the future. Dividends will not decrease unless companies expect long-run poorer prospects of the company in the future.
\end{remark}

\begin{remark} \hlt{Dividend Initiation}\\
Information conveyed is ambiguous. Dividend initiation may indicate that company is optimistic about its future and is sharing its wealth with stockholders; company may also have lack of profitable investment opportunities.
\end{remark}

\begin{remark} \hlt{Unexpected Dividend Increase}\\
Signal that business prospects are strong, managers will share success with shareholders.\\
Companies with long history of dividend increases are dominant in their industries, and have high return on assets and low debt ratios.
\end{remark}

\begin{remark} \hlt{Unexpected Dividend Decrease or Omissions}\\
Typically negative signals that business is in trouble, management does not believe current dividend payment can be maintained. May also mean profitable investment opportunities are available, shareholders will receive greater benefit by having earnings reinvested in company.
\end{remark}

\begin{remark} \hlt{Agency Costs and Dividends as Control Mechanism}
\begin{enumerate}[label=\roman*.]
\setlength{\itemsep}{0pt}
\item Between shareholders and managers: managers have incentive to over-invest, leading to investment in some negative NPV projects which reduces stockholder wealth. To reduce agency cost, increase payout of FCF as dividends. Mature firms in relatively non-cyclical industries do not need to hoard cash, hence a higher dividend payout would be welcomed by investors, resulting in increases in stock value.
\item Between shareholders and bondholders: when there is risky debt outstanding, shareholders may pay themselves a large dividend, leaving bondholders with lower asset base as collateral, hence resulting in wealth transfer from bondholders to stockholders. Typically resolved via provisions in bond indentures, which include restrictions on dividend payment, maintenance of certain BS ratios, etc.
\end{enumerate}
\end{remark}

\begin{definition} \hlt{Taxation Methods}
\begin{enumerate}[label=\roman*.]
\setlength{\itemsep}{0pt}
\item Double taxation system: corporate pretax earnings taxed at corporate levels, then taxed again at shareholder level if distributed to taxable shareholders as dividends.
\item Dividend imputation tax system: corporate earnings first taxed at corporate level. When earnings are distributed to shareholders as dividends, shareholders receive a tax credit (franking credit). If shareholder's marginal tax rate is higher than company's, shareholder pays the difference between the two rates.
\end{enumerate}
\end{definition}

\begin{remark} \hlt{Factors Affecting Dividend Payout Policy}
\begin{enumerate}[label=\roman*.]
\setlength{\itemsep}{0pt}
\item Investment opportunities: availability of positive NPV investment opportunities and speed at which firm must react to opportunities determine amount of cash firm must keep at hand. If there are many profitable opportunities and quick reaction required, dividend payout must be low.
\item Expected volatility of future earnings: firms tie target payout ratio to long-run sustainable earnings, are reluctant to increase dividends unless reversal is not expected in the near future.\\
When earnings are volatile, firms are more cautions in changing dividend payout.
\item Financial flexibility: firms with excess cash and desire to maintain financial flexibility may use stock repurchases instead of cash dividends, which is less sticky. Having cash on hand allows flexibility to meet unforeseen operating needs and investment opportunities, which is especially important in times of crisis where liquidity is low and credit is hard to obtain.
\item Tax considerations: depends on method and amount of tax applied on dividend payment.\\
For companies with favourable capital gains tax compared to dividends, high-tax-bracket investors prefer low dividend payouts, low-tax-bracket investors prefer high dividend payouts.\\
Lower dividend tax rate compared to capital gains do not mean companies will raise their dividend payouts. Stockholders may not prefer higher dividend payout as:
\begin{enumerate}[label=\arabic*.]
\setlength{\itemsep}{0pt}
\item taxes on dividends are paid when dividend is received, while capital gains taxes are paid only when shares are sold
\item cost basis of shares may receive step-up in valuation at shareholder's death, hence taxes on capital gains may not have to be paid at all
\item tax-exempt institutions will be indifferent between dividends or capital gains
\end{enumerate}
\item Flotation costs: when company issues new shares of common stock, flotation costs of $3\%$ to $7\%$ is taken from amount of capital raised to pay for costs associated with issuing new stock. As retained earnings have no such fee, cost of new equity capital is always higher than cost of retained earnings.\\
Larger companies have lower flotation costs as compared to smaller companies.\\
The higher the flotation costs, the lower the dividend payout.
\item Contractual and legal restrictions: Companies may be restricted from paying dividends by legal requirements, or by implicit restrictions due to cash needs of the business:
\begin{enumerate}[label=\arabic*.]
\setlength{\itemsep}{0pt}
\item Impairment of capital rule: dividends paid cannot be in excess of retained earnings
\item Debt covenants; designed to protect bondholders. May have target for liquidity ratios and coverage ratios before a dividend can be paid
\end{enumerate}
\end{enumerate}
\end{remark}


\begin{definition} \hlt{Taxation Methods}
\begin{enumerate}[label=\roman*.]
\setlength{\itemsep}{0pt}
\item Double taxation system: corporate pretax earnings taxed at corporate levels, then taxed again at shareholder level if distributed to taxable shareholders as dividends.
\begin{equation}
\text{Effective tax rate} = \text{Corporate tax rate} + (1-\text{Corporate tax rate})(\text{Individual tax rate}) \nonumber
\end{equation}
\item Split rate tax system: earnings distributed as dividends are taxes at lower rate than earnings retained. Effect is to offset the higher (double) tax rate applied to dividends at individual level.\\
Calculation of effective rate is similar to that under double taxation, except rate applicable would be corporate tax rate for distributed income
\item Dividend imputation tax system: taxes paid at corporate level but are attributed to shareholder, hence all taxes are effectively paid at shareholder rate.\\
Shareholders deduct their portion of taxes paid by corporation from their tax return. If shareholder tax bracket is lower than company rate, shareholder receive tax credit (franking credit) equal to the difference. If shareholder tax bracket is higher than company rate, shareholder pays the difference.\\
Effective tax rate on dividend is simply the shareholder's marginal tax rate.
\end{enumerate}
\end{definition}

\subsubsection{Payout Policies and Share Repurchases}

\begin{method} \hlt{Target Payout Adjustment Model}\\
Model of gradual adjustment from stable dividend payout policy to target dividend payout ratio.\\
If company earnings are expected to increase, and current payout ratio is below target payout ratio, investor may estimate future dividends with the following:
\begin{align}
\text{Expected increase in dividends} &= [(\text{Expected earnings} \times \text{Target payout ratio}) - \text{Previous dividend}]\times \text{Adj Fac} \nonumber \\
\text{Adj Fac} &= \frac{1}{\text{Number of years which adjustment in dividends will take place}} \nonumber
\end{align}
\end{method}

\begin{remark} \hlt{Trends in Dividend Payout Policy}
\begin{enumerate}[label=\roman*.]
\setlength{\itemsep}{0pt}
\item In developed markets, proportion of companies paying cash dividends trend downwards over LT.
\item Percentage of companies making stock repurchases are trending upwards in US since 1980s and in UK and EU since 1990s. Major companies in Asia (CN, JP), made substantial repurchases since 2010s.
\end{enumerate}
\end{remark}

\begin{remark} Share Repurchase Methods
\begin{enumerate}[label=\roman*.]
\setlength{\itemsep}{0pt}
\item \hlt{Open Market Transactions}: most flexible, allow company to buyback shares at most favourable terms. No obligation for company to complete an announced buyback program.\\
US companies do not need shareholder approval for open market transactions, unlike EU companies.
\item \hlt{Fixed-Price Tender Offer}: firm buys a predetermined number of shares at a fixed price (typically premium over current market price). Company forgoes flexibility to buy back shares quickly.\\
If more than desired number of shares tendered to offer, company will buyback prorated number of shares from each shareholder responding to the offer.
\item \hlt{Dutch Auction}: tender offer where company specifies a range of prices. Identify minimum clearing price for desired number of shares that need to be repurchased. Each participating shareholder indicates price and number of shares tendered. Bids accepted based on lowest price first until desired quantity is filled. Price of last offer accepted will be price paid for all shares tendered. Can be accomplished quickly, but not as quick as tender offers.
\item \hlt{Repurchase by Direct Negotiation}: purchase shares from a major shareholder with premium over market price. Used in greenmail scenario (hostile bidder is offered a premium to go away), also to remove a large overhang in the market that is dampening the share price. Many negotiated transactions occur at discount to market price, indicating urgent liquidity needs of the seller motivating the transaction.
\end{enumerate}
\end{remark}

\begin{remark} \hlt{Financial statement Effects of Repurchases}\\
Repurchases made with surplus cash will decrease cash and shareholder's equity, hence increasing leverage.\\
After repurchase, earnings per share may increase, depending on how much cash was used.\\
If repurchase was financed with additional debt offerings, reduction in net income from (after-tax) cost of borrowed funds to be factored to determine impact on earnings per share.\\
If price paid for share repurchase is higher (lower) than pre-repurchase book value per share (BVPS), then BVPS will decrease (increase).\\
If cost of capital is greater than earnings yield (earnings-to-price ratio), then earnings dilution will result from buyback. If earnings yield is greater than after-tax cost of borrowed funds, EPS will increase.
\end{remark}

\begin{remark} \hlt{Rationales for Share Repurchases}
\begin{enumerate}[label=\roman*.]
\setlength{\itemsep}{0pt}
\item Potential tax advantages: If tax rate on capital gains is lower than tax rate on dividend income, share repurchases have tax advantage over cash dividends.
\item Share price support/signalling: share repurchase signals to market that company views its own stock as a good investment and the future outlook is good, and is important in presence of asymmetric information. Tactic is often used when share price is declining, and management wants to convey confidence.
\item Added flexibility: as paying cash dividend and repurchasing shares are economically equivalent, company could declare small stable dividend, then repurchase shares with leftover earnings. Managers have discretion with respect to market timing of their repurchases.
\item Offsetting dilution from employee stock options: offset EPS dilution from exercise of employee options.
\item Increasing financial leverage: if funded by new debt. May change company's capital structure towards a more optimal one by decreasing the percentage of equity.
\end{enumerate}
\end{remark}

\begin{remark} \hlt{Dividend Safety}\\
Metric used to evaluate probability of dividends continuing at the current rate for a company.\\
Traditional ratios such as dividend payout ratio, or its inverse (dividend coverage ratio) are typically used for this purpose. A higher than normal payout ratio indicates a higher probability of dividend cut.\\
Compare the computed ratio to average ratio for the industry and market within which a company operates. Stable or increasing dividends are more favourable.\\
FCFE Coverage Ratio may also be considered, where dividends and share repurchases are both considered:
\begin{equation}
\text{FCFE Coverage Ratio} = \frac{\text{FCFE}}{\text{Dividends} + \text{Share repurchases}} \nonumber
\end{equation}
FCFE coverage ratio significantly less than one is considered unsustainable, as company is drawing down its cash reserves for dividends and repurchases.
\end{remark}