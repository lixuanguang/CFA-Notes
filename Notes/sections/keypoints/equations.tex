\subsection{Formula Sheet}

\subsubsection{Quantitative Methods: Multiple Regression}

ANOVA Table\\
\begin{tabularx}{\textwidth}{X|X|X|p{12em}|X}
\hline
\rowcolor{gray!30}
Source & Sum of Squares & $df$ & Mean Square & $F$-Statistics \\
\hline
Regression & $SSR$ & $k$ & $MSR = \frac{SSR}{k}$ & $F = \frac{MSR}{MSE}$ \\
Error & $SSE$ & $n-k-1$ & $MSE = \frac{SSE}{n-k-1} = SE^2$ & \\
Total & $SST$ & $n-1$ & &\\
\hline
\end{tabularx}

\begin{enumerate}[label=\roman*.]
\setlength{\itemsep}{0pt}
\item Logistic Regression: uses $log$-odds.
\begin{align}
\ln \frac{P}{1-P} &= \beta_0 + \beta_1 X_1 + \cdots + \beta_n X_n + \epsilon \nonumber \\
P &= \frac{1}{1 + \exp[-(\beta_0 + \beta_1 X_1 + \cdots + \beta_n X_n)]} \nonumber
\end{align}
\item $t$-value for Regression Coefficient: reject $H_0: \beta_i = 0$ if $t$-statistic < $t$-critical for two-tailed test, where $\beta_i$ is coefficient value, and $SE_i$ is standard error of coefficient.
\begin{equation}
t_{\beta_i} = \frac{\beta_i}{SE_i} \nonumber
\end{equation}
\item R-Squared: 
\begin{equation}
R^2 = \frac{SSR}{SST} \nonumber
\end{equation}
\item Adjusted R-Squared:
\begin{equation}
\overline{R}^2 = 1 - \left( \frac{n-1}{n-k-1} \right)(1-R^2) = 1 - \frac{SSE / (n-k-1)}{SST/(n-1)} \nonumber
\end{equation}
\item Restricted vs Unrestricted Model: If $F$-statistic $>$ $F$-value, unrestricted model is not significant.\\
Let $q$ be number of independent variable difference between $R, U$ model.\\
The test is one-tailed, right side, on $\alpha \%$.
\begin{equation}
F = \frac{(SSE_R - SSE_U)/q}{SSE_U/(n-k-1)}, \ \ \ df = (q, n-k-1) \nonumber
\end{equation}
\item Akaike's Information Criterion: lower means better prediction.
\begin{equation}
AIC = n \ln \left( \frac{SSE}{n} \right) + 2 (k+1) \nonumber
\end{equation}
\item Bayesian Information Criterion: lower means better fit.
\begin{equation}
BIC = n \ln \left( \frac{SSE}{n} \right) + \ln (n) (k+1) \nonumber
\end{equation}
\item Test for Conditional Heteroskedasticity: Breusch-Pagan (BP) Test. One-tailed test\\
If test statistic $>$ test critical, reject $H_0$. Hence evidence of heteroskedasticity.\\
Let $n$ be number of observations, $R^2$ be regression from BP test.
\begin{equation}
\text{BP Test Statistic} = \chi^2_{k} = nR^2, \ \ \ df = k \nonumber
\end{equation}
\item Test of Multi-Collinearity: Variance Inflation Factor (VIF). \\
\begin{align}
VIF_j = \frac{1}{1 - R^2_j} \nonumber
\end{align}
$VIF_j > 5$ warrants investigation.\\
$VIF_j > 10$ indicates serious multi-collinearity.
\item Test of Leverage: to detect points of extreme observations of independent variable.\\
If $h_{ii} > 3 (\frac{k+1}{n})$, it is considered potentially influential. $k$ is $\#$ independent variable.
\item Test of Outlier: Studentised Residuals. Extreme observations of dependent variable.
\begin{equation}
t_{i^*} = \frac{e^{*}_i}{s_{e^*}} = \frac{e_i}{\sqrt{MSE_{(i)} (1-h_{ii})}} \nonumber
\end{equation}
where $e^{*}_i$ is residual with $i$th observation deleted, $s_{e^*}$ is s.d. of all residuals, $k$ is number of independent variables, $MSE_{(i)}$ is MSE of regression model that deletes $i$th observation, $h_{ii}$ is $i$th observation leverage.\\
If $\abs{t_{i^*}} > 3$, flag observation as outlier.\\
If $\abs{t_{i^*}} >$ $t$ critical value with $df=n-k-2$, potential influential
\item Test of Influential Data Points: Cook's Distance. Points exclusion causes substantial changes in regression.\\
\begin{equation}
D_i = \frac{e_i^2}{(k+1)MSE} \left[ \frac{h_{ii}}{(1-h_{ii})^2} \right] \nonumber
\end{equation}
where $e_i$ is residual for observation $i$.\\
Cook's distance is distributed as an $F$-distribution with $df = (k+1, n-k-1)$.\\
If $D_i \geq 0.5$, observation may be influential and merits further investigation.\\
If $D_i \geq 1$ or $D_i \geq \sqrt{k/n}$, observation is highly likely to be an influential datapoint.
\end{enumerate}


\subsubsection{Quantitative Methods: Time Series Analysis}

\begin{enumerate}[label=\roman*.]
\setlength{\itemsep}{0pt}
\item Test for Serial Correlation: Breusch-Godfrey (BG) Test.\\
If test statistic $>$ test critical, reject $H_0$. Hence evidence of serial correlation.\\
$F$-test $df = (p, n-p-k-1)$ where $p$ is $\#$ lag, $n$ is $\#$ observation, $k$ is $\#$ independent variables.
\item Test for Serial Correlation: Durbin-Watson (DW) Test.\\
If test value $< d_L$, reject $H_0$. Hence evidence of positive serial correlation.\\
If test value $> d_U$, do not reject $H_0$.
\item Standard Error of Correlations: let $T$ be number of observations in regression.
\begin{equation}
SE = \frac{1}{\sqrt{T}} \nonumber
\end{equation}
\item Test of Autocorrelation of Residual: if $t$-statistic $>$ $\abs{t \text{ Critical}}$, residuals are serially correlated.
\begin{equation}
t = \frac{\rho_{\epsilon, k}}{SE}, \ \ \ df = T - 2 \nonumber
\end{equation}
\item 
\end{enumerate}