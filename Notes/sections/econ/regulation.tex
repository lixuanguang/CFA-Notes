\subsection{Economics of Regulation}

\begin{remark} \hlt{Economic Rationale for Regulation}\\
Required when markets cannot provide efficient solutions (Pareto optimal solution).
\begin{enumerate}[label=\roman*.]
\setlength{\itemsep}{0pt}
\item Information friction: lack of access to information and inadequate information. Result in adverse selection (private information with only some market participants), moral hazard (incentive conflicts). 
\item Externalities: spill over effects of production and consumption activities onto third parties.
\item Weak competition: results in high prices, less choices, lack of innovation.
\item Social objectives: resolved by provision of public goods (consumption by one does not reduce availability for others); through placing regulatory obligations on firm. Objectives include safety, privacy, protection, environmental, labour or employment, commerce or trade, financial systems.
\end{enumerate}
Regulation of financial markets increase confidence in the financial system, maintain integrity of markets, protection of consumers and investors. Prudential supervision, promote financial stability, reduce system-wide risks, protect customers.
\end{remark}

\begin{remark} \hlt{Regulation of Commerce}\\
Consumer protection, commercial law, antitrust sets out the framework for operation of private markets.\\
Government policy may be used for promoting commerce, such as establishing legal framework for contracting and setting standards, setting labour market rules and regulations, and on fundamental safety regulations.\\
Recognition and protection of intellectual properties, enforcement is a focus.\\
Technological change brings issue of data privacy and regulations such as GDPR.
\end{remark}

\begin{remark} \hlt{Regulation of Financial Markets}\\
Critical to prevent failures of financial system and to maintain integrity of markets.\\
Objectives include: protect investors, create confidence in markets, enhance capital formation.
\begin{enumerate}[label=\roman*.]
\setlength{\itemsep}{0pt}
\item Regulation of Security Markets: ensure fairness and integrity of capital markets, protect investors.
\begin{enumerate}[label=\arabic*.]
\setlength{\itemsep}{0pt}
\item Disclosure requirements provide transparency, promote investor confidence.
\item Securities regulations are directed toward mitigating agency problems.
\item Regulations have historically focused on protecting small (retail) investors .
\end{enumerate}
\item Regulation of Financial Institutions: \hlt{Prudential supervision} is the monitoring and regulation of financial institutions to reduce system-wide risks and protect investors. Focuses on diversification of assets, adequate capital base, and risk management activities of financial institutions.
\end{enumerate}
The cost benefit analysis of financial market regulations should also include hidden costs, i.e., FDIC insurance for banks may incentivise them with excessive risk-taking.
\end{remark}

\begin{remark} \hlt{Antitrust Regulation}\\
For promoting competition. If an M\&A is expected to reduce competition, regulators may block the M\&A.\\
Prohibit anticompetitive arrangements or practices such as price collusion, exchange of certain info, anticompetitive behaviours.
Companies need to satisfy simultaneously regulators across multiple jurisdictions.\\
When evaluating an announced M\&A, to consider anticipated response by regulators as part of the analysis.
\end{remark}

\begin{definition} \hlt{Types of Regulations}
\begin{enumerate}[label=\roman*.]
\setlength{\itemsep}{0pt}
\item Statutes: laws by legislative bodies
\item Administrative Regulations: rules issued by government agencies, bodies authorised by government
\item Judicial Law: findings of the court
\item Substantive law: focus on rights, responsibilities of entities, relationship between entities
\item Procedural law: focus on protection and enforcement of substantive laws.
\end{enumerate}
\end{definition}

\begin{definition} \hlt{Types of Regulators}\\
Regulatory bodies can be government departments and agencies or independent regulators who derive their power and authority from the state.
\begin{enumerate}[label=\roman*.]
\setlength{\itemsep}{0pt}
\item Independent regulators do not rely on govt funding, has some autonomy in decision making, thus take more technical and long-term view of policies.
\item Self-regulatory bodies (SRBs) are private organisations that both represent and regulate members. Derives authority from members. Entry requirements may be imposed.
\item Self-regulating organisations (SROs) are given recognition and authority, including enforcement power. Funded independently.
\end{enumerate}
\end{definition}

\begin{remark} \hlt{Uses of Self-Regulation in Financial Markets}\\
FINRA is an SRO recognised by SEC in US; primarily objective is to protect investors by maintaining fairness of US capital markets. FINRA has the authority to enforce security laws and regulations.\\
In civil-law countries, use of SROs is not common; formal government agencies fulfil role of SROs. Non-independent SRBs may support regulatory framework via guidelines, codes of conduct, and continuing education.\\
In common-law countries (UK, US), SROs have historically enjoyed recognition.
\end{remark}

\begin{definition} \hlt{Regulatory Capture}\\
Regulatory actions and determinations can restrict potential competition (i.e. limit entry), or effectively coordinate choices of rivals (i.e. impose certain quality standards or price controls).\\
Regulatory differences across jurisdictions may lead to:
\begin{enumerate}[label=\roman*.]
\setlength{\itemsep}{0pt}
\item Regulatory competition: regulators compete to provide the most business-friendly regulatory environment.
\item Regulatory arbitrage: businesses choose a country that allows a specific behaviour; entails exploiting the difference between economic substance and interpretation of regulation.
\end{enumerate}
Global regulatory cooperation and coordination is required to achieve cohesive regulatory framework.
\end{definition}

\begin{remark} \hlt{Tools of Regulatory Intervention}
\begin{enumerate}[label=\roman*.]
\setlength{\itemsep}{0pt}
\item Price mechanisms: taxes, subsidies to further specific regulatory objectives
\item Regulatory mandates and restrictions on behaviours: rights and responsibilities
\item provision of private goods, public financing of private project, bail-in tools (shareholders and creditors having their claims cancelled or reduced to extent necessary to restore institution to financial viability).
\end{enumerate}
Effectiveness of regulatory tools depends on enforcement abilities of regulators.\\
Enforcement should have desired effect of compliance with the regulations.
\end{remark}

\begin{remark} \hlt{Cost Benefit Analysis of Regulation}\\
Regulatory burden refers to the cost of regulation for the regulated entities.\\
Net regulatory burden refers to private costs less private benefits of regulation.\\
Regulators to be aware of unintended consequences of regulations. Regulatory burden is more difficult to measure as it includes indirect costs related to changes in economic behaviour.\\
Regulatory costs and benefits are difficult to assess on prospective basis relative to retrospective basis. Retrospective analysis allows more informed assessment of regulation as actual costs and benefits may be identifiable
\end{remark}

\begin{remark} \hlt{Regulatory Change Impact on Revenues}\\
May impact revenues if regulators introduce limits on prices, tariffs, rents, or fees that companies may charge; this is to protect consumers. Products may be banned, or required to provide descriptions to discourage consumption. Companies may find alternative ways to generate revenue.\\
If pricing or charging are not regulated or limited, companies may be required to increase pricing transparency and breakdown of fees. Natural monopolies may also be limited to a certain return on assets.
\end{remark}

\begin{remark} \hlt{Regulatory Change Impact on Costs}\\
Regulations may impact costs with higher operating expenses due to safety regulations, or due to labour laws which impose limits on hours of work.\\
Regulators may step in to require firms to pay compensations to customers, or ban certain activities.
\end{remark}

\begin{remark} \hlt{Considerations for Evaluating Effects of Regulations}\\
Regulations can help or hinder a company or industry. Regulations may shrink the size of one industry and increase size of another through taxation and subsidies. To review the impact of current and proposed regulations on an industry or company because regulations can have a large impact on valuation.\\
Regulations are not always costly for those that end up being regulated. If the regulator is captive, regulations may end up benefiting the regulated entities.\\
Regulations may introduce inefficiencies in the market. For example, past government bailouts of financial institutions have conveyed a message of future implicit guarantees.\\
Some regulations may be specifically applicable to certain sectors, while others may have broad implications affecting a number of sectors. Certain industries have more exposure to certain types of regulations. 
\end{remark}