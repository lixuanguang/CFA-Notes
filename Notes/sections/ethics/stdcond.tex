\subsection{Standards of Professional Conduct}

\begin{definition} \hlt{Standards of Professional Conduct}
\begin{enumerate}[label=\Roman*.]
\setlength{\itemsep}{0pt}
\item Professionalism
\begin{enumerate}[label=\Alph*.]
\setlength{\itemsep}{0pt}
\item Knowledge of the Law\\
Members and Candidates must understand and comply with all applicable laws, rules, and regulations (including the CFA Institute Code of Ethics and Standards of Professional Conduct) of any government, regulatory organization, licensing agency, or professional association governing their professional activities. In the event of conflict, Members and Candidates must comply with the more strict law, rule, or regulation. Members and Candidates must not knowingly participate or assist in and must dissociate from any violation of such laws, rules, or regulations.
\item Independence and Objectivity\\
Members and Candidates must use reasonable care and judgment to achieve and maintain independence and objectivity in their professional activities. Members and Candidates must not offer, solicit, or accept any gift, benefit, compensation, or consideration that reasonably could be expected to compromise their own or another’s independence and objectivity.
\item Misrepresentation\\
Members and Candidates must not knowingly make any misrepresentations relating to investment analysis, recommendations, actions, or other professional activities.
\item Misconduct\\
Members and Candidates must not engage in any professional conduct involving dishonesty, fraud, or deceit or commit any act that reflects adversely on their professional reputation, integrity, or competence.
\end{enumerate}
\item Integrity of Financial Markets
\begin{enumerate}[label=\Alph*.]
\setlength{\itemsep}{0pt}
\item Material Nonpublic Information\\
Members and Candidates who possess material nonpublic information that could affect the value of an investment must not act or cause others to act on the information.
\item Market Manipulation\\
Members and Candidates must not engage in practices that distort prices or artificially inflate trading volume with the intent to mislead market participants.
\end{enumerate}
\item Duties to Clients
\begin{enumerate}[label=\Alph*.]
\setlength{\itemsep}{0pt}
\item Loyalty, Prudence, and Care\\
Members and Candidates have a duty of loyalty to their clients and must act with reasonable care and exercise prudent judgment. Members and Candidates must act for the benefit of their clients and place their clients’ interests before their employer’s or their own interests.
\item Fair Dealing\\
Members and Candidates must deal fairly and objectively with all clients when providing investment analysis, making investment recommendations, taking investment action, or engaging in other professional activities.
\item Suitability
\begin{enumerate}[label=\arabic*.]
\setlength{\itemsep}{0pt}
\item When Members and Candidates are in an advisory relationship with a client, they must:
\begin{enumerate}[label=\alph*.]
\setlength{\itemsep}{0pt}
\item Make a reasonable inquiry into a client’s or prospective client’s investment experience, risk and return objectives, and financial constraints prior to making any investment recommendation or taking investment action and must reassess and update this information regularly.
\item Determine that an investment is suitable to the client’s financial situation and consistent with the client’s written objectives, mandates, and constraints before making an investment recommendation or taking investment action.
\item Judge the suitability of investments in the context of the client’s total portfolio.
\end{enumerate}
\item When Members and Candidates are responsible for managing a portfolio to a specific mandate, strategy, or style, they must make only investment recommendations or take only investment actions that are consistent with the stated objectives and constraints of the portfolio.
\end{enumerate}
\item Performance Presentation\\
When communicating investment performance information, Members and Candidates must make reasonable efforts to ensure that it is fair, accurate, and complete.
\item Preservation of Confidentiality\\
Members and Candidates must keep information about current, former, and prospective clients confidential unless:
\begin{enumerate}[label=\arabic*.]
\setlength{\itemsep}{0pt}
\item The information concerns illegal activities on the part of the client or prospective client,
\item Disclosure is required by law, or
\item The client or prospective client permits disclosure of the information.
\end{enumerate}
\end{enumerate}
\item Duties to Employers
\begin{enumerate}[label=\Alph*.]
\setlength{\itemsep}{0pt}
\item Loyalty\\
In matters related to their employment, Members and Candidates must act for the benefit of their employer and not deprive their employer of the advantage of their skills and abilities, divulge confidential information, or otherwise cause harm to their employer.
\item Additional Compensation Agreements\\
Members and Candidates must not accept gifts, benefits, compensation, or consideration that competes with or might reasonably be expected to create a conflict of interest with their employer’s interest unless they obtain written consent from all parties involved.
\item Responsibilities of Supervisors\\
Members and Candidates must make reasonable efforts to ensure that anyone subject to their supervision or authority complies with applicable laws, rules, regulations, and the Code and Standards.
\end{enumerate}
\item Investment Analysis, Recommendations, and Actions
\begin{enumerate}[label=\Alph*.]
\setlength{\itemsep}{0pt}
\item Diligence and Reasonable Basis\\
Members and Candidates must:
\begin{enumerate}[label=\arabic*.]
\setlength{\itemsep}{0pt}
\item Exercise diligence, independence, and thoroughness in analysing investments, making investment recommendations, and taking investment actions.
\item Have a reasonable and adequate basis, supported by appropriate research and investigation, for any investment analysis, recommendation, or action.
\end{enumerate}
\item Communication with Clients and Prospective Clients\\
Members and Candidates must:
\begin{enumerate}[label=\arabic*.]
\setlength{\itemsep}{0pt}
\item Disclose to clients and prospective clients the basic format and general principles of the investment processes they use to analyse investments, select securities, and construct portfolios and must promptly disclose any changes that might materially affect those processes.
\item Disclose to clients and prospective clients significant limitations and risks associated with the investment process.
\item Use reasonable judgment in identifying which factors are important to their investment analyses, recommendations, or actions and include those factors in communications with clients and prospective clients.
\item Distinguish between fact \& opinion in presentation of investment analysis and recommendations.
\end{enumerate}
\item Record Retention\\
Members and Candidates must develop and maintain appropriate records to support their investment analyses, recommendations, actions, and other investment-related communications with clients and prospective clients.
\end{enumerate}
\item Conflicts of Interest
\begin{enumerate}[label=\Alph*.]
\setlength{\itemsep}{0pt}
\item Disclosure of Conflicts\\
Members and Candidates must make full and fair disclosure of all matters that could reasonably be expected to impair their independence and objectivity or interfere with respective duties to their clients, prospective clients, and employer. Members and Candidates must ensure that such disclosures are prominent, are delivered in plain language, and communicate the relevant information effectively.
\item Priority of Transactions\\
Investment transactions for clients and employers must have priority over investment transactions in which a Member or Candidate is the beneficial owner.
\item Referral Fees\\
Members and Candidates must disclose to their employer, clients, and prospective clients, as appropriate, any compensation, consideration, or benefit received from or paid to others for the recommendation of products or services.
\end{enumerate}
\item Responsibilities as a CFA Institute Member or CFA Candidate
\begin{enumerate}[label=\Alph*.]
\setlength{\itemsep}{0pt}
\item Conduct as Participants in CFA Institute (CFAI) Programs\\
Members and Candidates must not engage in any conduct that compromises the reputation or integrity of CFAI or the CFA designation or the integrity, validity, or security of CFAI programs.
\item Reference to CFA Institute, the CFA Designation, and the CFA Program\\
When referring to CFAI, CFAI membership, the CFA designation, or candidacy in the CFA Program, Members and Candidates must not misrepresent or exaggerate the meaning or implications of membership in CFAI, holding the CFA designation, or candidacy in the CFA Program.
\end{enumerate}
\end{enumerate}
\end{definition}

\subsubsection{Guidance for Standard I(A): Knowledge of the Law}

\begin{remark} \hlt{Code and Standards vs Local Law}
\begin{enumerate}[label=\roman*.]
\setlength{\itemsep}{0pt}
\item Must know the laws and regulations relating to professional activities in all countries business is conducted.
\item Always adhere to the most strict rules and requirements (CFAI Standards or the law) that apply
\item Must comply with applicable laws or regulations related to their professional activities.
\item Must not engage in conduct that constitutes a violation of the Code and Standards, even though it may otherwise be legal.
\end{enumerate}
\end{remark}

\begin{remark} \hlt{Participation or Association with Violations by Others}
\begin{enumerate}[label=\roman*.]
\setlength{\itemsep}{0pt}
\item To dissociate or separate from any ongoing client or employee activity that is illegal or unethical, even if it involves leaving an employer (in an extreme case).
\item Attempt to stop the behaviour by bringing this to attention of supervisor or compliance first.
\item Inaction with continued association may be construed as participation or assistance.
\item To consult legal and compliance advisers for guidance; do not compel reporting of violations to government or regulatory organisations unless such disclosure is mandatory under applicable law.
\end{enumerate}
\end{remark}

\begin{remark} \hlt{Investment Product and Applicable Laws}
\begin{enumerate}[label=\roman*.]
\setlength{\itemsep}{0pt}
\item To review whether associated firms that are distributing products or services developed by their employing firm also abide by the laws and regulations of the countries and regions of distribution
\item To undertake the necessary due diligence when transacting cross-border business to understand the multiple applicable laws and regulations in order to protect the reputation of their firm and themselves.
\end{enumerate}
\end{remark}

\begin{flushleft}
NS: countries wth no security laws or regulations\\
LS: countries with less strict securities laws and regulations than Code and Standards
MS: countries with more strict securities laws and regulations than Code and Standards
\begin{tabularx}{\textwidth}{X|p{14em}}
\hline
\rowcolor{gray!30}
Applicable Law & Duties\\
\hline
Reside in NS country, business in LS country; LS law applies & Adhere to code and standards\\
\hline
Reside in NS country, business in MS country; MS law applies & Adhere to MS country\\
\hline
Reside in LS country, business in NS country; LS law applies & Adhere to code and standards\\
\hline
Reside in LS country, business in MS country; MS law applies & Adhere to MS country\\
\hline
Reside in LS country, business in NS country; LS law applies, but states that law of locality where business is conducted governs & Adhere to code and standards\\
\hline
Reside in LS country, business in MS country; LS law applies, but states that law of locality where business is conducted governs & Adhere to MS country\\
\hline
Reside in MS country, business in LS country; MS law applies & Adhere to MS country\\
\hline
Reside in MS country, business in LS country; MS law applies, but states that law of locality where business is conducted governs & Adhere to code and standards\\
\hline
Reside in MS country, business in LS country with client who is citizen of LS country; MS law applies, but states that law of client country governs & Adhere to code and standards\\
\hline
Reside in MS country, business in LS country with client who is citizen of MS country; MS law applies, but states that law of client country governs & Adhere to MS country\\
\hline
\end{tabularx}
\end{flushleft}

\begin{remark} \hlt{Recommended Procedures: Members and Candidates}
\begin{enumerate}[label=\roman*.]
\setlength{\itemsep}{0pt}
\item To have procedures to keep up with changes in applicable laws, rules, and regulations.
\item To review compliance procedures on a regular basis to ensure that the procedures reflect current law, CFAI Standards and regulations.
\item To maintain current reference materials for employees to access to keep up to date.
\item To seek advice of counsel or compliance department when in doubt.
\item To document any violations when they disassociate themselves from prohibited activity and encourage employers to bring an end to such activity.
\item No requirement under Standards to report violations to governmental authorities, but may be advisable in some circumstances and required by law in others.
\item Strongly encouraged to report other member's violations of the Code and Standards
\end{enumerate}
Members who supervise creation and maintenance of investment services and products to be aware of and comply with regulations and laws regarding such services and products both in country of origin and where they will be sold.
\end{remark}

\begin{remark} \hlt{Recommended Procedures: Firms}
\begin{enumerate}[label=\roman*.]
\setlength{\itemsep}{0pt}
\item To develop and/or adopt a code of ethics
\item Make available to employees information that highlights applicable laws and regulations
\item Establish written procedures for reporting suspected violations of laws, regulations, or company policies
\end{enumerate}
\end{remark}

\subsubsection{Guidance for Standard I(B): Independence and Objectivity}

\begin{remark} \hlt{Investments, Gifts}
\begin{enumerate}[label=\roman*.]
\setlength{\itemsep}{0pt}
\item Do not let investment process to be influenced by any external sources. 
\item Modest gifts are permitted.
\item Allocation of shares in oversubscribed IPOs to personal accounts is not permitted.
\item Distinguish between gifts from clients and gifts from entities seeking to influence to detriment of client.
\item Gifts must be disclosed to employer in any case, either prior to acceptance if possible, or subsequently.
\end{enumerate}
\end{remark} 

\begin{remark} \hlt{Investment Banking Relationships}
\begin{enumerate}[label=\roman*.]
\setlength{\itemsep}{0pt}
\item Do not be pressured to issue favourable research on current or prospective investment-banking clients.
\item Only appropriate for analyst to work with investment bankers in road shows when conflicts are adequately and effectively managed and disclosed.
\item Ensure there are effective firewalls between research, investment management, and IB activities.
\end{enumerate}
\end{remark}

\begin{remark} \hlt{Public Companies}
\begin{enumerate}[label=\roman*.]
\setlength{\itemsep}{0pt}
\item Analysts should not be pressured to issue favourable research by companies they follow.
\item Do not confine research to discussions with company management.
\item Use variety of sources in research, including suppliers, customers, and competitors.
\end{enumerate}
\end{remark}

\begin{remark} \hlt{Buy-Side Clients}
\begin{enumerate}[label=\roman*.]
\setlength{\itemsep}{0pt}
\item Buy-side clients may try to pressure sell-side analysts. Portfolio managers may have large positions in particular security, and rating downgrade may have effect on portfolio performance.
\item Portfolio manager to ensure respect and foster intellectual honesty of sell-side research.
\end{enumerate}
\end{remark}

\begin{remark} \hlt{Fund Manager and Custodial Relationships}
Members responsible for selecting outside managers should not accept gifts, entertainment, or travel that might be perceived as impairing their objectivity.
\end{remark}

\begin{remark} \hlt{Performance Measurement and Attribution}
Performance analysts may experience pressure from investment managers who have produced poor results or acted outside their mandate. Must not let such influences affect their analysis.
\end{remark}

\begin{remark} \hlt{Manager Selection}
\begin{enumerate}[label=\roman*.]
\setlength{\itemsep}{0pt}
\item To exercise independence and objectivity when selecting investment managers.
\item Must not accept gifts or other compensation that could be seen as influencing hiring decisions, nor offer compensation when seeking to be hired as investment managers.
\item Responsibility to maintain independence and objectivity applies to all hiring and firing decisions, not just those that involve investment management.
\end{enumerate}
\end{remark}

\begin{remark} \hlt{Credit Rating Agencies}
\begin{enumerate}[label=\roman*.]
\setlength{\itemsep}{0pt}
\item Members in credit rating firms to ensure that procedures prevent undue influence by security issuer.
\item Members who use credit ratings should be aware of this potential conflict and interest, and consider whether independent analysis is warranted.
\end{enumerate}
\end{remark}

\begin{remark} \hlt{Issuer-Paid Research}\\
Fraught with potential conflicts. Analysts' compensation for preparing such research should be limited, and preference is for a flat fee, without regard to conclusions or report recommendations.
\end{remark}

\begin{remark} \hlt{Travel}\\
Best practice is for analysts to pay for their own commercial travel when attending information events or tours sponsored by the firm being analysed.
\end{remark}

\begin{remark} \hlt{Recommended Procedures}
\begin{enumerate}[label=\roman*.]
\setlength{\itemsep}{0pt}
\item Protect the integrity of opinions, make sure they are unbiased.
\item Create a restricted list and distribute only factual information about companies on the list
\item Restrict special cost arrangements. Pay for one's own commercial transportation and hotel; limit use of corporate aircraft to cases in which commercial transportation is not available.
\item Limit gifts to token gifts only. Customary, business-related entertainment is fine as long as its purpose is not to influence a member's professional independence or objectivity. Impose clear value limits on gifts.
\item Restrict employee investments in IPOs and private placements. Require pre-approval of IPO purchases.
\item Review procedures, have effective supervisory and review procedures.
\item Firms should have formal written policies on independence and objectivity of research.
\item Firms should appoint a compliance officer and provide clear procedures for employee reporting of unethical behaviour and violations of applicable regulations.
\end{enumerate}
\end{remark}

\subsubsection{Guidance for Standard I(C): Misrepresentation}

\begin{remark} \hlt{Guidance}
\begin{enumerate}[label=\roman*.]
\setlength{\itemsep}{0pt}
\item Do not make any misrepresentations or give false impressions. This includes oral, electronic, social media communications. Misrepresentations include guaranteeing investment performance and plagiarism (using someone else's work without giving credit).
\item Knowingly omitting information that could affect an investment decision or performance evaluation is considered misrepresentation.
\item Models and analysis developed by others in member firm are firm property, can be used without attribution. Report written by another analyst cannot be released as another analyst's work.
\end{enumerate}
\end{remark}

\begin{remark} \hlt{Recommended Procedures for Compliance}
\begin{enumerate}[label=\roman*.]
\setlength{\itemsep}{0pt}
\item Firm to provide employees who deal with clients or prospects a written list of firm's available services and a description of firm's qualifications. Employee qualification to be accurately presented as well.
\item To avoid plagiarism, maintain records of all materials used to generate reports or other firm products, properly cite sources (quotes and summaries) in work products. Information from recognised financial and statistical reporting services need not be cited.
\item Firms to establish procedures for verifying marketing claims of third parties whose information the firm provide to clients.
\end{enumerate}
\end{remark}

\subsubsection{Guidance for Standard I(D): Misconduct}

\begin{remark} \hlt{Guidance}\\
CFAI discourages unethical behaviour in all aspects of members' and candidates' lives.\\
Do not abuses CFAI Professional Conduct Program by seeking enforcement of this Standard to settle personal, political, or other disputes that are not related to professional ethics.
\end{remark}

\begin{remark} \hlt{Recommended Procedures for Compliance}\\
Firms are encouraged to adopt the following:
\begin{enumerate}[label=\roman*.]
\setlength{\itemsep}{0pt}
\item Develop and adopt code of ethics and make clear unethical behaviour will not be tolerated.
\item Give employees a list of potential violations and sanctions, including dismissal.
\item Check references of potential employees.
\end{enumerate}
\end{remark}

\subsubsection{Guidance for Standard II(A): Material Nonpublic Information}

\begin{remark} \hlt{Material Information}\\
Information is material if its disclosure would impact price of a security or if reasonable investors would want the information before making an investment decision.\\
The more ambiguous the effect of the information on price, the less material that information is considered.\\
Information is non-public until it has been made available to the marketplace. Analyst conference call is not public disclosure. Selectively disclosing information creates the potential for insider-trade violations.\\
Prohibition against against on material nonpublic information extends to mutual funds containing the subject securities as well related swaps and option contracts.\\
Nonpublic information may be used for its intended purpose (i.e., investment banking transactions), but must not use the information for any other purpose unless it becomes public information.
\end{remark}

\begin{remark} \hlt{Mosaic Theory}\\
There is no violation when analyst reaches an investment conclusion about a corporate action or event through analysis of public information together with nonmaterial nonpublic information.
\end{remark}

\begin{remark} \hlt{Social Media}\\
When gathering information from internet or social media sources, not all is considered public information.\\
To confirm that any material information received from these sources is also available from public sources.
\end{remark}

\begin{remark} \hlt{Industry Experts}\\
May seek insight from individuals who have specialised expertise in an industry. However, to not act or cause others to act on any material nonpublic information obtained until that information is made public.
\end{remark}

\begin{remark} \hlt{Investment Research Reports}\\
When a well-known analyst issues a report or makes changes to recommendations, the information alone may have effect on the market, thus be considered material.\\
Simply because public would find the conclusions material does not require analyst make work public. Investors who are not clients can either do the work themselves or become clients of the analysts to gain insights.
\end{remark}

\begin{remark} \hlt{Recommended Procedures for Compliance}
\begin{enumerate}[label=\roman*.]
\setlength{\itemsep}{0pt}
\item Make reasonable efforts to achieve public dissemination of the information
\item Firms to adopt compliance procedures to prevent misuse of material nonpublic information.
\item Firms to adopt disclosure policies designed to ensure information is disseminated to marketplace in an equitable manner. Company to not discriminate among analysts in provision of information.
\item To issue press release prior to analyst meetings and conference calls, script the meetings and calls. If material nonpublic information is disclosed for first time in call, to issue a press release or otherwise make the information publicly available.
\item Use of firewall within the firm, with elements including:
\begin{enumerate}[label=\arabic*.]
\setlength{\itemsep}{0pt}
\item substantial control of relevant interdepartmental communications, either through a clearance area such as compliance or the legal department
\item review of employee trading through maintenance of 'watch', 'restricted', 'rumour' lists
\item documentation of procedures designed to limit flow of information between departments, and actions taken to enforce those procedures
\item heightened review or restriction of proprietary trading while a firm is in possession of material non-public information
\end{enumerate}
\item Prohibition of all proprietary trading while firm is in possession of material nonpublic information may be inappropriate as it may send a signal to the market. In this case, firms to take only the contra side of unsolicited customer trades.
\end{enumerate}
\end{remark}

\subsubsection{Guidance for Standard II(B): Market Manipulation}

\begin{remark} \hlt{Guidance}
\begin{enumerate}[label=\roman*.]
\setlength{\itemsep}{0pt}
\item Information-based manipulation includes spreading false rumours to induce trading by others.
\item Transaction-based manipulation includes transactions that artificially affect prices or volume, and securing a controlling, dominant position in a financial instrument to exploit and manipulate the price of a related derivative and/or underlying asset.
\end{enumerate}
\end{remark}

\subsubsection{Guidance for Standard III(A): Loyalty, Prudence, and Care}

\begin{remark} \hlt{Identification of Actual Investment Client}\\
Client might be an individual, beneficiaries of pension plans or trusts.\\
If actual client or group of beneficiaries may not exist, to invest in manner consistent with stated mandate.\\
The 'client' may be the investing public as a whole rather than a specific entity or person.
\end{remark}

\begin{remark} \hlt{Guidance}\\
Client interests always come first, to act in client's best interest and recommend products that are suitable given client's investment objectives and risk tolerances.
\begin{enumerate}[label=\roman*.]
\setlength{\itemsep}{0pt}
\item Exercise the prudence, care, skill, and diligence under the circumstances that a person acting in a like capacity and familiar with such matters would use.
\item Manage pools of client assets in accordance with terms of governing documents, such as trust documents, or investment management agreements.
\item Make investment decision sin context of total portfolio.
\item Inform clients of any limitations in an advisory relationship (i.e., advisor recommend own firm products)
\item Vote proxies in an informed and responsible manner. Due to cost benefit considerations, it may not be necessary to vote all proxies.
\item Client brokerage ('soft dollars', 'soft commissions') must be used to benefit the client.
\end{enumerate}
\end{remark}

\begin{remark} \hlt{Recommended Procedures for Compliance}\\
Submit to clients itemised statements showing all securities in custody, all debits, credits, transactions at least quarterly. Encourage firms to address these topics when drafting policies and procedures on fiduciary duty:
\begin{enumerate}[label=\roman*.]
\setlength{\itemsep}{0pt}
\item Follow all applicable rules and laws
\item Establish the investment objectives of the client. Consider suitability of portfolio relative to client needs and circumstances, the investment basic characteristics, or basic characteristics of the total portfolio
\item Diversity investments to reduce risk of loss, unless contrary to account objectives
\item Carry out regular reviews
\item Deal fairly with all clients with respect to investment actions.
\item Disclose conflicts of interest
\item Disclose compensation arrangements
\item Vote proxies in best interest of clients and ultimate beneficiaries
\item Maintain confidentiality
\item Seek best execution
\item Place client interests first
\end{enumerate}
\end{remark}

\subsubsection{Guidance for Standard III(B): Fair Dealing}

\begin{remark} \hlt{Discrimination}
\begin{enumerate}[label=\roman*.]
\setlength{\itemsep}{0pt}
\item Do not discriminate against any clients when disseminating recommendations or taking investment action. 
\item Fairly does not mean equally. In normal course of business, there will be differences in the time message is received by clients. 
\item Different service levels is fine; disclose different service levels to all clients and prospects, and make premium levels of service available to all who wish to pay for them.
\end{enumerate}
\end{remark}

\begin{remark} \hlt{Investment Recommendations}
\begin{enumerate}[label=\roman*.]
\setlength{\itemsep}{0pt}
\item Give all clients a fair opportunity to act upon every recommendation.
\item Clients who are unaware of change in recommendation should be advised before order is accepted.
\end{enumerate}
\end{remark}

\begin{remark} \hlt{Investment Actions}\\
Treat clients fairly in light of their investment objectives and circumstances. Treat individual and institutional clients in a fair and impartial manner.\\
Should not take advantage of position in industry to disadvantage clients (i.e., during IPOs).
\end{remark}

\begin{remark} \hlt{Recommended Procedures for Compliance}\\
Encourage firms to establish compliance procedures requiring proper dissemination of investment recommendations and fair treatment of all customers and clients. Compliance procedures may be established as such:
\begin{enumerate}[label=\roman*.]
\setlength{\itemsep}{0pt}
\item Limit the number of people who are aware that a change in recommendation will be made
\item Shorten the time frame between decision and dissemination
\item Publish personal guidelines for pre-dissemination. Have in place guidelines prohibiting personnel who have prior knowledge of a recommendation from discussing it or taking action on the pending recommendation
\item Simultaneous dissemination of new or changed recommendation to all clients who have expressed an interest or for whom an investment is suitable
\item Maintain list of clients and holdings, ensure that all holders are treated fairly
\item Develop written trade allocation procedures, which ensure fairness to clients, timely and efficient order execution, and accuracy of client positions
\item Disclose trade allocation procedures
\item Establish systematic account review. Ensure that no client is given preferred treatment and that investment actions are consistent with account's objectives
\item Disclose available levels of service.
\end{enumerate}
\end{remark}

\subsubsection{Guidance for Standard III(C): Suitability}

\begin{remark} \hlt{General Guidance}\\
In advisory relationships, gather client information in the form of an investment policy statement (IPS).\\
Consider client needs and circumstances, and thus their risk tolerance, whether use of leverage is suitable.\\
If responsible for managing a fund to an index or other stated mandate, ensure consistency with mandate.
\end{remark}

\begin{remark} \hlt{Unsolicited Trade Requests}\\
If client request for trades that do not properly align with IPS, the trade may or may not have a material effect on the risk characteristics of client total portfolio and requirements are different for each case.\\
In either case, manager should not make the trade until discussion with the client the reasons (based on IPS) that the trade is unsuitable for the client's account.\\
If effect on risk/return profile on client total portfolio is minimal, after discussion, may follow firm's policy; client must acknowledge the discussion and an understanding of why the trade is unsuitable.\\
If effect on risk/return profile is material, one option is to update the IPS so client accepts a changed risk profile that would permit the trade. If client does not accept a changed IPS, follow firm policy, which may allow the trade to be made in a separate client-directed account. Otherwise, reconsider relationship with client.
\end{remark}

\begin{remark} \hlt{Recommended Procedures for Compliance}
\begin{enumerate}[label=\roman*.]
\setlength{\itemsep}{0pt}
\item Put each client's needs, circumstances, investment objectives into a written IPS.
\item Consider type of client, and whether there are separate beneficiaries, investor objectives (risk and return), investor constraints (liquidity needs, expected cash flows, time, tax, regulatory and legal circumstances), and performance measurement benchmarks.
\item Review investor objectives and constraints periodically to reflect any changes in client circumstances
\end{enumerate}
\end{remark}

\subsubsection{Guidance for Standard III(D): Performance Presentation}

\begin{remark} \hlt{Guidance}
\begin{enumerate}[label=\roman*.]
\setlength{\itemsep}{0pt}
\item Avoid misstating performance or misleading clients/prospects about investment performance
\item Should not misrepresent past performance or reasonably expected performance
\item Should not state or imply the ability to achieve a rate of return similar to that achieved in the past
\item For brief presentations, to make detailed information available on request, and indicate that the presentation has offered limited information.
\end{enumerate}
\end{remark}

\begin{remark} \hlt{Recommended Procedures for Compliance}
\begin{enumerate}[label=\roman*.]
\setlength{\itemsep}{0pt}
\item Encourage firms to adhere to Global Investment Performance Standards (GIPS)
\item Consider the sophistication of audience to whom a performance presentation is addressed
\item Presenting performance of weighted composite of similar portfolios rather than single account
\item Include terminated accounts as part of historical performance, clearly stating when they were terminated
\item Include all appropriate disclosures to fully explain results (i.e., model results included, gross or net of fees)
\item Maintain data and records used to calculate the performance being presented
\end{enumerate}
\end{remark}

\subsubsection{Guidance for Standard III(E): Preservation of Confidentiality}

\begin{remark} \hlt{Guidance}
\begin{enumerate}[label=\roman*.]
\setlength{\itemsep}{0pt}
\item If client is involved in illegal activities, members have obligation to report activities to authorities. The confidentiality standard extends to former clients as well.
\item Requirements of this standard is not intended to prevent members from cooperating with a CFAI Professional Conduct Program (PCP) investigation.
\end{enumerate}
\end{remark}

\begin{remark} \hlt{Recommended Procedures for Compliance}
\begin{enumerate}[label=\roman*.]
\setlength{\itemsep}{0pt}
\item To avoid disclosing information received from client except to authorised co-workers who are also working for the client.
\item To follow firm procedures for storage of electronic data and recommend adoption of such procedures if they are not in place.
\end{enumerate}
\end{remark}

\subsubsection{Guidance for Standard IV(A): Loyalty}

