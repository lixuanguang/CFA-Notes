\subsection{Basic Financial Statement Analysis}

\subsubsection{Income Statement Analysis}

\begin{remark} \hlt{Revenue Recognition}\\
Revenue is recognised even if cash is not collected until next accounting period.
\end{remark}

\begin{definition} \hlt{Common Income Statement Line Items}
\begin{enumerate}[label=\roman*.]
\setlength{\itemsep}{0pt}
\item Net Sales: 
\begin{equation}
\text{Net Sales} = \text{Gross Sales} - \text{Sales returns and allowances} - \text{discounts} \nonumber
\end{equation}
\item Gross Margin: 
\begin{equation}
\text{Gross Margin} = \text{Net Sales} - \text{Cost of Goods Sold} \nonumber
\end{equation}
\item Operating Expenses: expenses other than cost of goods sold (COGS), i.e. selling expenses, general and administrative expenses
\item Operating Income: 
\begin{equation}
\text{Operating Income} = \text{Gross Margin} - \text{Operating Expenses} \nonumber
\end{equation}
\item Earnings Before Interest and Taxes (EBIT): amount earned from all activities before income taxes.
\item Net Income: gross margin.
\begin{equation}
\text{Net Income} = \text{EBIT} - \text{Income Tax} \nonumber
\end{equation}
\end{enumerate}
\end{definition}

\begin{method} \hlt{B.A.S.E. Technique}
\begin{enumerate}[label=\roman*.]
\setlength{\itemsep}{0pt}
\item[B:] Beginning balance
\item[A:] Add cash payments and liability account ending balances
\item[S:] subtract asset accounting ending balances
\item[E:] equals ending balance
\end{enumerate}
\end{method}

\begin{definition} \hlt{Accrual Accounting}\\
Revenue is recognised when earned.\\
If revenue is on credit, it is on trade and accounts receivable account.\\
If revenue is earned in advance, there is liability account for unearned revenue.
\end{definition}

\begin{method} \hlt{Revenue Recognition}
\begin{enumerate}[label=\arabic*.]
\setlength{\itemsep}{0pt}
\item Identify the contract with a customer
\item Identify distinct performance obligations in the contract
\item Determine contract transaction price
\item Allocate transaction price to obligations
\item Recognize revenue when obligation is satisfied.
\end{enumerate}
\end{method}

\begin{remark} \hlt{Revenue Recognition Conditions}
\begin{enumerate}[label=\roman*.]
\setlength{\itemsep}{0pt}
\item Completion of earnings process (no obligation for future services, i.e. warranty protection)
\item Assurance of payment (quantified amount must be reliable)
\end{enumerate}
\end{remark}

\begin{method} \hlt{Percentage of Completion Method}
\begin{align}
\text{Percentage completed} &= \frac{\text{Costs incurred to date}}{\text{Most recent estimate of total costs}} \nonumber \\
\text{Revenue to be recognised to-date} &= \text{Percent completed} \times \text{Estimated total revenue} \nonumber \\
\text{Current period revenue} &= \text{Revenue to be recognised to-date} - \text{Revenue recognised prior} \nonumber
\end{align}
\end{method}

\begin{method} \hlt{Completed Contract Method}\\
Used if there is no contract, or estimates are unreliable, or ability to collect revenue is uncertain.
\end{method}

\begin{method} \hlt{Instalment Sales Method}\\
Used if COGS are known, but collectability of sale proceeds cannot be reasonably determined.
\begin{align}
\text{Gross Profit Rate} &= \frac{\text{Sales} - \text{COGS}}{\text{Sales}} \nonumber \\
\text{Realised Gross Profit} &= \text{Cash Collection} \times \text{Gross Profit Rate} \nonumber
\end{align}
\end{method}

\begin{method} \hlt{Cost Recovery Method}\\
More conservative than instalment sales. Used if COGS cannot be reasonably determined.\\
Sales recognised when cash is received but no gross profit is recognised until all of COGS collected. Profit recognised only when cash collections exceed total COGS.
\end{method}

\begin{method} \hlt{Barter Transaction Recognition}\\
Revenue should be reported only if fair value of transaction is determined based on company’s historical practice of receiving cash for similar transaction from buyers unrelated to the counterparty for the barter.
\end{method}

\begin{method} \hlt{Reseller Revenue Recognition}
\begin{enumerate}[label=\roman*.]
\setlength{\itemsep}{0pt}
\item Gross reporting used if company has general inventory risk, can determine product price, can change supplier, bears credit risk. 
\item Net reporting if company is sales agent.
\end{enumerate}
\end{method}

\begin{definition} \hlt{Revenue Matching Principle}
\begin{enumerate}[label=\roman*.]
\setlength{\itemsep}{0pt}
\item Operating expenses only recognised when the work or product makes contribution to revenue.
\item Expenses are to be grouped by either function or nature.
\item Current period expenses to appear on Income Statement.
\item Future period expenses are capitalised. When revenues are recognised, asset is converted to expenses in these periods.
\end{enumerate}
\end{definition}

\begin{method} \hlt{Direct Write-Off Method}\\
Uncollectible accounts charged to expense in the period they are determined to be worthless.\\
Revenue matching principle is not adhered to.
\end{method}

\begin{method} \hlt{Allowance Method}\\
Bad debt expense recorded in same period as sale.\\
Estimate on percentage-of-sales basis (on IS) or outstanding receivables (on BS) basis.
\end{method}

\begin{method} \hlt{Warranty Recognition}\\
Recognize estimate warranty expense in period of the sale, and update expense indicated by experience over life of warranty.
\end{method}
