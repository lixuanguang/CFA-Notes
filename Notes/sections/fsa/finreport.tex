\subsection{Financial Reporting}

\begin{remark} Role of Financial Statement Analysis
\begin{enumerate}[label=\roman*.]
\setlength{\itemsep}{0pt}
\item Role of financial reporting: provide information about company’s financial position for use by internal and external parties.
\item Role of financial analysis: evaluate company past, current, prospective financial position and performance for investment, credit, and similar decisions
\end{enumerate}
\end{remark}

\begin{definition} \hlt{Financial Statements}
\begin{enumerate}[label=\roman*.]
\setlength{\itemsep}{0pt}
\item Balance sheet (BS): provides information on liquidity, solvency, financial position at a point of time.
\begin{equation}
\text{Assets} = \text{Liabilities} + \text{Owner's Equity} \nonumber
\end{equation}
\item Income statement (IS): provides information on financial performance of activities over period of time on a consolidated basis
\item Cash Flow Statement (CFS): discloses sources and use of cash. For liquidity, solvency, financial flexibility
\item Statement of changes in equity: shows changes in owner’s investment in the business over time, in order of liquidation and dividends
\item Financial footnotes: includes accounting methods (assumptions and estimates), and disclosure on fixed assets, inventory methods, income taxes, pensions, debt, contingencies etc.
\item Supplementary schedules: includes additional info on assets and liabilities of company, but is unaudited
\item Management commentary: includes specific issues on financial statements, current financial condition, liquidity, and planned capital expenditure (Capex). Not audited, for public companies only.
\end{enumerate}
\end{definition}

\begin{remark} \hlt{Auditor Reports}. In accordance with GAAP, identify inconsistent principles.
\begin{enumerate}[label=\roman*.]
\setlength{\itemsep}{0pt}
\item Unqualified opinion: free of material misstatements (by GAAP). Fairly represented.
\item Qualified opinion: 1 to 2 situations not compliant with GAAP, rest are fairly presented.
\item Adverse opinion: materially misstated, generally do not comply with GAAP. Unreliable, inaccurate.
\item Disclaimer of opinion: auditor could not form and refuses to present an opinion. Issued when auditor cannot complete work.
\end{enumerate}
\end{remark}

\begin{remark} \hlt{Standard Setting Bodies}. These are private sector, self-regulated bodies.
\begin{enumerate}[label=\roman*.]
\setlength{\itemsep}{0pt}
\item IASB: Standard-setting body of IFRS Foundation. Deliberate, develop, issue international financial reporting standards.
\item FASB: Issues new and revised standards to develop standards of financial reporting.  US GAAP recognised by SEC, but SEC retains authority to establish standards.
\end{enumerate}
Principles: To provide full, accurate, and timely disclosure of financial results, risks, and other information material to investor’s decisions. High and internationally acceptable quality.
\end{remark}

\begin{remark} \hlt{Regulatory Bodies}. These have the legal authority to enforce financial reporting requirements, can overrule private-sector standard setting bodies.
\begin{enumerate}[label=\roman*.]
\setlength{\itemsep}{0pt}
\item IOSCO: Regulate world financial markets. Protect investors, ensure markets are fair, efficient, and transparent, and reduce systematic risk.
\item SEC: Governs form and content of financial statements through securities act. Oversees PCAOB.
\end{enumerate}
\end{remark}

\begin{remark} \hlt{Key Regulations}
\begin{enumerate}[label=\roman*.]
\setlength{\itemsep}{0pt}
\item Securities exchange act of 1934: Created SEC, give SEC authority over all aspects of securities industry, empower SEC to require periodic reporting.
\item Securities act of 1993: Specified financial and other significant information that investors must receive when securities are sold, prohibits misrepresentations, requires initial registration of all public issuances of securities.
\item Sarbanes-Oxley Act of 2002: Oversee auditors. Ensure auditor independence, corporate responsibility for financial reports, effectiveness of firm’s internal control over financial reporting.
\end{enumerate}
\end{remark}

\begin{flushleft}
Financial Statement Analysis Process: \\
\begin{tabularx}{\textwidth}{c|p{90pt}XX}
\hline
\rowcolor{gray!30}
Step & Step Name & Input & Output \\
\hline
$1$ & Articulate the purpose and context of the analysis & 
\xxx The nature of the analyst's function and context of the analysis such as evaluating an equity or debt investment or issuing a credit rating.
\xxx Communication with client or supervisor on needs and concerns.
\xxx Institutional guidelines related to developing specific work product.
& 
\xxx Statement of the purpose or objective of the analysis.
\xxx A list (written or unwritten) of specific questions to be answered by the analysis.
\xxx Nature and content of the report to be provided.
\xxx Timetable and budgeted resources for completion. \\
\hline
$2$ & Collect input data
& 
\xxx Financial statements, other financial data, questionnaires, and industry, economic data.
\xxx Discussions with management, suppliers, customers, and competitors.
\xxx Company site visits (e.g., to production facilities or retail stores).
& 
\xxx Organised financial statements.
\xxx Financial data tables.
\xxx Completed questionnaires, if applicable. \\
\hline
$3$ & Process data
&
\xxx Data from previous phase.
&
\xxx Adjusted financial statements.
\xxx Common-size statements.
\xxx Ratios and graphs.
\xxx Forecasts \\
\hline
$4$ & Analyse and interpret the processed data
&
\xxx Input data as well as processed data.
&
\xxx Analytical results. \\
\hline
$5$ & Develop and communicate conclusions and recommendations (e.g., with an analysis report). &
\xxx Analytical results and previous reports.
\xxx Institutional guidelines for published reports.
&
\xxx Analytical report answering questions posed in Phase 1.
\xxx Recommendation regarding the purpose of the analysis, such as whether to make an investment or grant credit. \\
\hline
$6$ & Follow up &
\xxx Information gathered by periodically repeating above steps as necessary to determine whether changes to holdings or recommendations are necessary. &
\xxx Updated reports and recommendations. \\
\hline
\end{tabularx}
\end{flushleft}

\begin{remark} \hlt{Types of Reports}
\begin{enumerate}[label=\roman*.]
\setlength{\itemsep}{0pt}
\item Registration statement: provides disclosure about securities offered for sale; relationship of new securities to other securities; informational provided in annual filings; recent audited financial statement; risk factors in the business.
\item Forms 10-K, 20-F, 40-F: Forms 10-K are for US registrants, 40-F are for Canadian, and 20-F for other non-US registrations. This is a legal document with minimal marketing. Provides information on business, financial disclosures, legal proceedings, information related to management. 
\item Annual report: Not SEC requirement. Opportunity for company to present itself to stakeholders and other external parties. Highly polished marketing document. Overlap with 10-K.
\item Proxy statements, Form DEF-14A: Provides information on litigation, executive compensation, related-party transactions. Proposals that require shareholder vote, security ownership by management and principal owners, director’s biographic information.
\item Interim reports, Forms 10-Q, 6-K: Provided on a quarterly basis, less detailed than annual reports, unaudited statements and footnotes. If no-recurring events take place, included in 10-Q report.
\item Forms 8-K: Announce major events such as acquisitions, disposal of corporate assets, changes in securities and trading markets, matters related to accountants and financial statements, corporate governance and management changes, regulation FD disclosures.
\item Forms 3, 4, 5: Report beneficial ownership of securities for any owners greater than 10\% per class of securities. Form 3 is initial statement, Form 4 is changes, Form 5 is annual report.
\item Form 155: Notice of proposed sale of restricted securities or securities held by affiliate of the issuer.
\item Form 11-K: Annual report of employee stock purchase, savings, etc.
\end{enumerate}
\end{remark}

\begin{definition} \hlt{Financial Reporting Recognition Principles}
\begin{enumerate}[label=\roman*.]
\setlength{\itemsep}{0pt}
\item \hlt{Probable}: economic outcome has high probability of occurrence.
\item \hlt{Measurable}: economic outcome measured exactly with reliability.
\end{enumerate}
\end{definition}

\begin{definition} \hlt{Financial Reporting Fundamental Qualitative Factors}
\begin{enumerate}[label=\roman*.]
\setlength{\itemsep}{0pt}
\item \hlt{Relevance}: potential to affect or make difference in user’s decisions. Predictive, confirmatory value.
\item \hlt{Materiality}: omission or misstatement can influence user decisions
\item \hlt{Faithful Representation}: complete, neutral, free from error
\end{enumerate}
\end{definition}

\begin{definition} \hlt{Financial Reporting Enhancing Qualitative Factors}\\
Comparable and consistent, verifiable, timeliness, and understandable.
\end{definition}

As it takes time to get reliable information, will need to get balance between relevance and reliability.

\begin{definition} \hlt{Accounting Assumptions}: on an accrual basis, going concern principle.
\end{definition}

\begin{definition} \hlt{Types of Costs}
\begin{enumerate}[label=\roman*.]
\setlength{\itemsep}{0pt}
\item \hlt{Historical Cost}: recorded at value paid at time of acquisition for assets, and liabilities proceeds in return for obligation.
\item \hlt{Amortised Cost}: historical cost adjusted for amortisation, depreciation, or depletion/impairment.
\item \hlt{Current Cost}: cash or cash equivalents if asset is paid for or liability required to settle obligation currently.
\end{enumerate}
\end{definition}

\begin{definition} \hlt{Types of Value}
\begin{enumerate}[label=\roman*.]
\setlength{\itemsep}{0pt}
\item \hlt{Realisable Value}: cash or cash equivalents if assets sold in an orderly disposal, and liability at settlement.
\item \hlt{Present Value}: assets at present value (PV) discounted of future cash flows. Liabilities at PV discounted of future net cash flows required to settle.
\item \hlt{Fair Value}: amount which an asset could be exchanged, or liability settled between willing parties.
\end{enumerate}
\end{definition}

\begin{remark} \hlt{IFRS Reporting Requirements}
\begin{enumerate}[label=\roman*.]
\setlength{\itemsep}{0pt}
\item Required financial statements: balance sheet, income statement, statement of changes in equity, cash flow statement, notes.
\item Required features: fair representation, going concern, accrual basis, consistency, materiality and aggregation, no offsetting. Annual frequency of reporting. Comparative information from previous periods.
\item Structure and Content:
\begin{enumerate}[label=\arabic*.]
\setlength{\itemsep}{0pt}
\item Balance Sheet: disclose current and non-current assets and liabilities, unless if liquidity-based presentation is more reliable and relevant.
\item Financial Statements: minimum line-item disclosures.
\item Notes: disclosures on information.
\item Comparative information: disclosed for previous period.
\end{enumerate}
\item Disclosure of accounting policies:
\begin{enumerate}[label=\arabic*.]
\setlength{\itemsep}{0pt}
\item Measurement bases used in preparing financial statements
\item Significant accounting policies used
\item Judgments made in applying accounting policies that have the most significant effect on the amounts recognised in the financial statements
\end{enumerate}
\item Sources of estimation uncertainty: Key assumptions about the future and other key sources of estimation uncertainty that have a significant risk of causing material adjustment to the carrying amount of assets and liabilities within the next year
\item Other Disclosures: 
\begin{enumerate}[label=\arabic*.]
\setlength{\itemsep}{0pt}
\item Information about capital and about certain financial instruments classified as equity
\item Dividends not recognised as a distribution during the period, including dividends declared before the financial statements were issued and any cumulative preference dividends
\item Description of the entity, including its domicile, legal form, country of incorporation, and registered office or business address
\item Nature of operations and principal activities
\item Name of parent and ultimate parent
\end{enumerate}
\end{enumerate}
\end{remark}

Effective financial reporting have the following characteristics: transparency, comprehensiveness, consistency

\begin{remark} \hlt{Barriers to a single standard}:
\begin{enumerate}[label=\roman*.]
\setlength{\itemsep}{0pt}
\item Valuation approach: judgement is required
\item Standard-setting approach: principles-based vs rule-based
\item Measurement approach: what constitutes an asset and a liability. Use of matching principle 
\end{enumerate}
\end{remark}

If new products are launched by a business, understand the business purposes, then evaluate potential effect on financial statements.
