\subsection{Analysis of Financial Institutions}

\begin{remark} \hlt{Financial Institutions Characteristics}
\begin{enumerate}[label=\roman*.]
\setlength{\itemsep}{0pt}
\item Systematic Importance: necessary for smooth functioning and overall health of the economy. As an intermediary between providers and users of capital, this creates inter-dependences that will introduce a system-wide failure if one institution fails (contagion effect). Bank deposits are insured up to a certain limit by the government to avoid financial contagion and reduce the risk of bank runs.
\item Regulated: financial institutions are highly regulated, with minimum capital requirements, minimum liquidity requirements, and limits on risk-taking.
\item Assets: assets are financial assets such as loans and securities that are usually reported at fair value.
\end{enumerate}
\end{remark}

\begin{remark} \hlt{Basel III Framework}:
\begin{enumerate}[label=\roman*.]
\setlength{\itemsep}{0pt}
\item Minimum required capital for a bank based on risk of bank's assets. The riskier the bank assets, the higher the required capital.
\item Minimum liquid assets to meet demands under a 30-day liquidity stress scenario.
\item Require stable funding relative to bank's liquidity needs over a one-year time horizon. Stability in funding is proportional to tenor of bank's deposits; longer-term deposits are more stable than shorter-term deposits. Stability also depends on the type of deposit.
\end{enumerate}
Basel III prompted banks to focus on asset quality, hold capital against other types of risk (i.e, operational risk), develop risk assessment processes.
\end{remark}

\subsubsection{CAMELS Approach}

\begin{definition} \hlt{Bank Capital Tiers}
\begin{enumerate}[label=\roman*.]
\setlength{\itemsep}{0pt}
\item Tier 1 Capital
\begin{enumerate}[label=\arabic*.]
\setlength{\itemsep}{0pt}
\item Common Tier 1 Capital: common stock, additional paid-in capital, issuance surplus related to common stock, retained earnings, OCI less intangibles and deferred tax assets.
\item Other Tier 1 Capital: subordinated instruments with no specific maturity and no contractual dividends (i.e., preferred stock with discretionary dividends).
\end{enumerate}
\item Tier 2 Capital: subordinated instruments with original (when issued) maturity of more than five years.
\end{enumerate}
Tier 1 plus Tier 2 capital makes up total capital of the bank.
\end{definition}

\begin{definition} \hlt{Capital Adequacy}\\
Proportion of bank assets funded with capital, adjusted based on risk (riskier assets have heavier weighting). Risk weighting specified by individual country regulators with Basel III.\\
Basel III guidelines specify the following:
\begin{enumerate}[label=\roman*.]
\setlength{\itemsep}{0pt}
\item Common Tier 1 Capital $\geq 4.5\%$ of risk-weighted assets
\item Total Tier 1 Capital $\geq 6.0\%$ of risk-weighted assets
\item Total Capital (Tier 1 + Tier 2) $\geq 8.0\%$ of risk-weighted assets
\end{enumerate}
\end{definition}

\begin{definition} \hlt{Asset Quality} \\
Assess amount of existing and potential credit risk associated with bank assets, focusing on financial assets.\\
Loans asset quality depends on creditworthiness of borrowers and corresponding adequacy of adjustments for expected loan losses. Measured at amortised cost, shown on BS net of allowances for loan losses.
\begin{flushleft}
\begin{tabularx}{\textwidth}{p{5em}|X|X}
\hline
\rowcolor{gray!30}
 & Equity & Debt \\
\hline
IFRS &
\xxx Fair value through OCI
\xxx Fair value through PnL &
\xxx Amortised cost
\xxx FVOCI
\xxx Fair value through PnL \\
\hline
GAAP &
\xxx Fair value through PnL &
\xxx Amortised cost (held-to-maturity)
\xxx FVOCI (available for sale)
\xxx Fair value through PnL (trading) \\
\hline
\end{tabularx}
\end{flushleft}
\end{definition}

\begin{remark} \hlt{Asset Quality: Credit Risk}
\begin{enumerate}[label=\roman*.]
\setlength{\itemsep}{0pt}
\item Off-balance sheet trading activities create exposure to counterparty credit risk.
\item Off-balance sheet obligations such as guarantees, unused committed credit lines, LOC create credit risk.
\end{enumerate}
\end{remark}

\begin{remark} \hlt{Asset Quality: Loan Loss Provisions}\\
'Allowance for Loan Losses' is a contra asset account to loans, the result of 'Provision for Loan Losses', an expense subject to management discretion. To evaluate bank policy of setting aside adequate provisions relative to actual loan performance. Actual losses are then written off these provisions.\\
Useful ratios for evaluation (that compares discretionary metric to more objective measure) are:
\begin{enumerate}[label=\roman*.]
\setlength{\itemsep}{0pt}
\item Ratio of allowance for loan losses to nonperforming loans
\item Ratio of allowance for loan losses to net loan charge-offs
\item Ratio of provision for loan losses to net loan charget-offs
\end{enumerate}
\end{remark}

\begin{definition} \hlt{Management Capabilities}\\
Risk management and internal control is critical for banks.\\
To look out for strong governance structure, sound internal controls, transparent management communication, financial reporting quality. Management should be able to identify and control risk, including credit risk, market risk, operating risk, legal risk, and other risks.
\end{definition}

\begin{definition} \hlt{Earnings}\\
Financial institutions should provide adequate return on capital, reward stockholders through capital appreciation and/or distribution of earnings. Look for high quality earnings from sustainable items.\\
Earnings estimates are based on the following:
\begin{enumerate}[label=\roman*.]
\setlength{\itemsep}{0pt}
\item Loan impairment allowances: assessments on likelihood of borrower default or bankruptcy, value of the collateral. Sensitive to risk factors such as economic and credit conditions across geographies.
\item Financial assets and liabilities valuation with fair value hierarchy.
\item Other areas common with non-financial companies, such as goodwill impairment, deferred tax asset, liability to recognise in connection with contingencies.
\end{enumerate}
Examine composition of earnings, which comprise of net interest income, service income, trading income (by most to least sustainable). Highly volatile net interest income may indicate excessive interest rate risk exposure.
\end{definition}

\begin{definition} \hlt{Fair Value Hierarchy}
\begin{enumerate}[label=\roman*.]
\setlength{\itemsep}{0pt}
\item Level 1: quoted prices for identical financial assets and liabilities in active markets
\item Level 2: quoted prices for similar financial instruments in active markets, quoted prices for identical financial instruments in non-active markets, observable data such as interest rates, yield curves, credit spreads, implied volatility. Used to model fair value of financial instrument.
\item Level 3: fair value based on model and unobservable inputs. More subjective.
\end{enumerate}
\end{definition}

\begin{definition} \hlt{Maturity Transformation}\\
Banks create value by borrowing money on shorter terms than terms for lending to customers. May destroy value if markets for short-term funding has a dislocation, or yield curve unexpectedly inverts.
\end{definition}

\begin{definition} \hlt{Liquidity Position}\\
Adequate liquidity is crucial for a bank. Basel III has two minimum liquidity standards:
\begin{enumerate}[label=\roman*.]
\setlength{\itemsep}{0pt}
\item Liquidity Coverage Ratio (LCR): 
\begin{equation}
\text{LCR} = \frac{\text{Highly Liquid Assets}}{\text{Expected Cash Outflows}} \geq 100\% \nonumber
\end{equation}
Highly liquid assets are those that are easily convertible into cash.\\
Expected cash flows are the estimated one-month liquidity needs in a stress scenario.
\item Net Stable Funding Ratio (NSFR):
\begin{equation}
\text{NSR} = \frac{\text{Available Stable Funding}}{\text{Required Stable Funding}} \geq 100\% \nonumber
\end{equation}
Available stable funding (ASF) is a function of the composition and maturity distribution of bank funding sources (i.e., capital, deposits, and other liabilities).\\
Required stable funding is a function of the composition and maturity distribution of bank asset base.\\
NSFR relates liquidity needs of bank assets to liquidity provided by bank liabilities (i.e., funding sources).
\end{enumerate}
Other liquidity monitoring metrics recommended by Basel III includes:
\begin{enumerate}[label=\roman*.]
\setlength{\itemsep}{0pt}
\item Concentration of Funding: proportion of funding obtained from single source. Lack of diversification may pose a problem when the sources withdraw funding, resulting in heightened liquidity risk for the bank.
\item Maturity Mismatch: when asset maturities differ materially from maturity of liabilities. The higher the mismatch, the higher the liquidity risk for the bank, which may expose the bank to a liquidity crunch if it is unable to roll over its borrowings at reasonable rates.
\end{enumerate}
\end{definition}

\begin{flushleft}
\begin{tabularx}{\textwidth}{p{40em}|X}
\hline
\rowcolor{gray!30}
Funding Component of ASF & ASF Factor \\
\hline
\xxx Total regulatory capital (exclude Tier 2 instruments maturing in a year)
\xxx Other capital instruments and liabilities with residual maturity $> 1$ year &
100\% \\
\hline
\xxx Stable demand deposits and term deposits with residual maturity $< 1$ year from retail and small business customers &
95\% \\
\hline
\xxx Less stable demand deposits and term deposits with residual maturity $< 1$ year from retail and small business owners &
90\% \\
\hline
\xxx Funding from non-financial corporate customers, sovereign, public sector, multilateral, national development banks with residual maturity $< 1$ year
\xxx Operational deposits
\xxx Other funding with residual maturity $> 6$ months and $< 1$ year not included in above categories, including funding from central banks and financial institutions &
50\% \\
\hline
\xxx All other liabilities not included in above categories, including liabilities without stated maturity (specific treatment for deferred tax liabilities, minority interests)
\xxx NSFR derivative liabilities net of NSFR derivative assets (if NSFR derivative liabilities > NSFR derivative assets)
\xxx 'Trade date' payables from purchase of fin instruments, foreign currencies, commodities &
0\% \\
\hline
\end{tabularx}
\end{flushleft}

\begin{definition} \hlt{Sensitivity to Market Risk}\\
Exposure to changes in interest rates, exchange rates, equity prices, or commodity prices.\\
Mismatches in maturity, repricing frequency, reference rates, or currency of bank loans and deposits create exposure to market movements.\\
Value at risk (VaR may be used to measure and monitor market risk.
\end{definition}

\subsubsection{Non-CAMELS Factors}

\begin{remark} \hlt{Government Support}\\
Larger banks have higher probability of implicit government support due to risk of contagion effect.\\
Government agencies will close banks that might fail, or arrange mergers with healthy ones to absorb them.\\
Factors include size of bank, status of country’s banking system (capacity to absorb single bank failure).
\end{remark}

\begin{remark} \hlt{Government Ownership}\\
Public ownership increases faith of implicit government backing in a bank.\\
Governments may aid financial development of banks, leading to broad economic growth.
\end{remark}

\begin{remark} \hlt{Mission of Banking Entity}\\
Community banks may by guided by community development in their lending decisions.\\
If community dependent on primary industry, may lead to concentration of risk in bank asset portfolio.
\end{remark}

\begin{remark} \hlt{Risk Factors}\\
Fill gaps on legal and regulatory issues, present in annual filing.
\end{remark}

\begin{remark} \hlt{Basel III Disclosures}\\
Provide regulatory information on consistent, comparable basis
\end{remark}

\begin{remark} \hlt{Corporate Culture}\\
Culture evaluation can be conducted by a review of:
\begin{enumerate}[label=\roman*.]
\setlength{\itemsep}{0pt}
\item Diversity of bank assets. If losses generated due to narrow investment strategy, then bank is too aggressive.
\item Accounting restatements due to failures of internal controls indicate unethical culture.
\item Excessive management compensation tied to bank stock performance may lead to excessive risk-taking.
\item Speed with which bank adjust loan loss provisions relative to actual loss behaviour. Slower response rate indicates aggressive accounting practices and a risk-taking culture.
\end{enumerate}
General factors relevant to analysis includes:
\begin{enumerate}[label=\roman*.]
\setlength{\itemsep}{0pt}
\item Competitive environment: global banks may take excessive risks to outdo large rivals.
\item Off-balance-sheet assets and/or liabilities may be opaque. Look for VIEs and SPEs.
\end{enumerate}
\end{remark}

\begin{remark} \hlt{Segment Information}\\
Segment information provide insights into different lines of business and geographies.\\Help investor decide whether capital is being allocated well within bank’s internally competing operations.
\end{remark}

\begin{remark} \hlt{Currency Exposure}\\
Significant for large, global banks trading in currencies or holding significant assets or liabilities in different currencies whose values fluctuate. Volatility in currency values may have significant impact on bank earnings.
\end{remark}

\begin{remark} \hlt{Risk Factors}\\
Fill gaps on legal and regulatory issues, present in annual filing.
\end{remark}

\begin{remark} \hlt{Basel III Disclosures}\\
Provide regulatory information on consistent, comparable basis
\end{remark}

\subsubsection{Insurance Companies}

Insurance has smaller proportion of cross-broader business. Insurance foreign branch required to hold assets in jurisdiction that are adequate to cover policy liabilities.\\
Insurance earn revenues from premium and from investment income earned on float.

\begin{remark} \hlt{Properties of P\&C Insurance Companies}\\
Premium income is the highest source of income. To diversity risk, insurers will reinsure some risks.\\
Policy period is very short, with premiums received at beginning of period and invested during float period.\\
Claim events are clearly defined, but may take a long time to emerge.\\
Property insurance covers protection on auto, homes, and specific assets. Casualty insurance protects against a legal liability due to occurrence of a covered event. Multiple peril policy covers both property and casualty.
\end{remark}

\begin{remark} \hlt{P\&C Insurer Profitability}\\
Business is cyclical and price-sensitive. Price cutting drive out profitability, competition lessens and underwriting standards tighten (hard pricing period), premiums rise and insurers return to more reasonable levels of profitability (soft pricing period), attracting more entrants; cycle repeats.\\
Expenses include claim expense, and expense of obtaining new policy business. Direct-to-customer model has fixed cost of staffing, and agency model has variable commissions.
\end{remark}

\begin{remark} \hlt{P\&C Insurer Combined Ratio}\\
Soft or hard pricing is driven by industry combined ratio. When ratio is low (high), it is a hard (soft) market.
\begin{align}
\text{Combined Ratio} &= \frac{\text{Insurance Expenses}}{\text{Net Premiums Earned}}\nonumber \\
&= \text{Underwriting Loss Ratio} + \text{Underwriting Expense Ratio} \nonumber
\end{align}
For single insurer, combined ratio $> 100\%$ indicates an underwriting loss.\\
The combined ratio is the sum of underwriting loss ratio and expense ratio.
\begin{align}
\text{Underwriting Loss Ratio} &= \frac{\text{Claims paid} + \Delta \text{Loss reserves}}{\text{Net premiums earned}} \nonumber \\
\text{Underwriting Expense Ratio} &= \frac{\text{Underwriting expenses including commissions}}{\text{Net premium written}} \nonumber
\end{align}
Underwriting loss ratio measures relative efficiency of company's underwriting standards (if policies are priced appropriately relative to risks borne). Lower is better.\\
Underwriting expense ratio measures efficiency of company operations. Lower is better.\\
\hlt{Loss reserve} is an estimated value of unpaid claims, subject to management discretion in management. Downward revisions indicate conservative loss estimation. Upward revision indicates aggressive profit booking.
\end{remark}

\begin{remark} \hlt{P\&C Insurer Other Profitability and Cost Ratios}
\begin{enumerate}[label=\roman*.]
\setlength{\itemsep}{0pt}
\item Loss and loss adjustment expense ratio: measure success in estimation of risk insured. Lower is better.
\begin{equation}
\text{Loss and loss adjustment expense ratio} = \frac{\text{Loss expense} + \text{Loss adjustment expense}}{\text{Net premiums earned}} \nonumber
\end{equation}
\item Dividends to policyholders (shareholders) ratio: liquidity measure of cash outflow on account of dividends relative to premium income
\begin{equation}
\text{Dividends to policyholders ratio} = \frac{\text{Dividends to policyholders (shareholders)}}{\text{Net premiums earned}} \nonumber
\end{equation}
\item Combined ratio after dividends (CRAD) measures total efficiency, takes into account cash satisfaction of policyholders or shareholders after consideration of total underwriting efforts.
\begin{align}
\text{Combined ratio} &= \text{Loss and loss adjustment expense ratio} + \text{Underwriting expense ratio} \nonumber \\
\text{CRAD} &= \text{Combined ratio} + \text{Dividends to policyholders ratio} \nonumber
\end{align}
\item Industry specific cost ratios include:
\begin{align}
&\frac{\text{Total benefits paid}}{\text{Net premiums written and deposits}} \nonumber \\
&\frac{\text{Commissions and expenses}}{\text{Net premiums written and deposits}} \nonumber
\end{align}
\end{enumerate}
\end{remark}

\begin{remark} \hlt{P\&C Insurer Investment Characteristics}\\
Investment preferred in steady-return, low-risk assets. Low-liquid assets shunned.\\
Concentration of assets by type, maturity, credit quality, industry, or geographical location or within single issuers should be evaluated.
\begin{equation}
\text{Total investment return ratio} = \frac{\text{Total investment income}}{\text{Invested assets}} \nonumber
\end{equation}
Computing the ratio after excluding unrealised capital gains from income provides information on importance of unrealised gains and losses to insurer's total income.
\end{remark}

\begin{remark} \hlt{P\&C Insurer Liquidity Considerations}\\
Liquidity is important for P\&C insurers as they stand ready to meet claim obligations.\\
To gauge liquidity of investment portfolio, look at fair value hierarchy reporting.
\end{remark}

\begin{remark} \hlt{Properties of L\&H Insurance Companies}\\
Premium income is the highest source of income. To diversity risk, insurers will reinsure some risks.\\
Life insurance policies can be basic term-life (insurer makes payment if death occurs during policy period).\\
Other policy types include investment products attached to pure life policies.
\end{remark}

\begin{remark} \hlt{L\&H Insurer Profitability}\\
Proportion of income from premiums, investments, and fees can vary over time and among insurers. Diversification is desirable, and premium income tends to be more stable over time relative to other sources.\\
Actuarial assumptions affect value of future liabilities due to policyholders; current period claim expense includes claim payments and interest on estimated liability to policyholders.\\
L\&H insurers capitalise cost of acquiring new and renewal policies and amortise it based on actual and estimated future profits from that business. Estimates influence amount amortised in any given period. Estimates also affect value of securities and investment returns.\\
Mismatches between valuation approaches for assets and liabilities can distort values when interest rate changes.
\end{remark}

\begin{remark} \hlt{L\&H Insurer Investment Characteristics}\\
L\&H insurers have longer float period, hence investment returns are key component to profitability.\\
Large portion of investment portfolio is LT debt; duration mismatch between assets and liabilities is of concern.\\
Similar to P\&C insurers, total investment income ratio is used to evaluate investment performance.
\end{remark}

\begin{remark} \hlt{L\&H Insurer Liquidity Considerations}\\
Policy surrenders can be unpredictable, but liquidity needs are fairly predictable.\\
Liquidity measure takes ratio of investment assets (adjusted based on ready convertibility to cash) to obligations (adjusted based on assumptions about withdrawals).\\
Ratio is estimated under both normal market conditions and under stress conditions.\\
Current ratio not directly applicable, as BS do not include such classifications.
\end{remark}

\begin{remark} \hlt{Insurer Capitalisation Regulations}\\
No global risk-based capital requirement standard for insurers.\\
EU has adopted Solvency II standards.\\
NAIC in United States has minimum capital levels based on size and risk.
\end{remark}
